\section{Dane}
\label{sec:dane}

%p-p, 13 TeV - Pythia8 - JJ events anchored to LHC16i, ALIROOT-7268

Dane użyte w analizie pochodzą z symulacji Monte Carlo zderzeń proton-proton przy energii w układzie środka masy równej $\sqrt{s} = 13$ TeV dostępnych na serwerach eksperymentu ALICE.
Są to pełne symulacje detektora ALICE, wykorzystujące generator zderzeń \texttt{Pythia8} \cite{Sjostrand:2007gs} (\textit{tune: Pythia8Jets\_Monash2013} \cite{Skands:2014pea}) oraz pakiet \texttt{Geant3} \cite{Brun:1994aa} do transportu cząstek przez materiał detektora.

Do rekonstrukcji dżetów wykorzystany został algorytm \textit{anti-kt} z parametrem $R = 0.4$ zaimplementowany w pakiecie \texttt{FASTJET} \cite{Cacciari:2011ma}. Dżetów poszukiwano wyłącznie wśród cząstek naładowanych \angterm{charged jets} ze względu na słabe pokrycie przestrzeni fazowej przez kalorymetry w eksperymencie ALICE.
%Symulacje prowadzone były w wąskich binach $p_T$, które zostały połączone.

Do analizy wybrano dżety o $p_T$ większym niż 15 GeV i mieszczące się w całości w akceptancji detektora \textit{TPC}, tj. $|\eta| < 0.9$, co przy użytym parametrze rozmiaru dżetu $R = 0.4$, daje ograniczenie na pseudopospieszność $|\eta| < 0.5$ dla osi dżetu.

\begin{figure}[h]
	\centering
	\includegraphics[width=0.5\textwidth]{sv-ip-def.png}
	\caption{Rysunek ilustrujący znaczenie używanych wielkości: $L_{xy}$ oraz parametru zderzenia $IP$. \imgsrc{\cite{sv-def-d0exp}}.}
	\label{fig:sv-ip-def}
\end{figure}

Dla każdego dżetu obliczony został szereg wielkości, które można podzielić na zmienne na poziomie dżetu, związane z wtórnymi wierzchołkami oraz cząstkami tworzącymi dżet. 
Za potencjalne wtórne wierzchołki uznaje się wszystkie kombinacje trzech cząstek spełniających pewne dosyć luźne kryteria jak $p_T > 0.15$ GeV (rozważane są wyłącznie trzy-cząstkowe wtórne wierzchołki), stąd ich liczba może być dużo większa od liczby cząstek tworzących dżet.


Lista używanych zmiennych:
\begin{itemize}
	\item Zmienne na poziomie dżetu:
	\begin{itemize}	
		\item $\eta$, $\phi$ -- pseudopospieszność i kąt azymutalny osi dżetu
		\item $p_T$ -- pęd poprzeczny dżetu
		\item masa dżetu [\#REF]
		\item powierzchnia dżetu -- liczona w płaszczyźnie ($\eta$, $\phi$),   do powierzchni dżetu zaliczany jest element w powierzchni w którym dodanie cząstki o nieskończenie małym pędzie poprzecznym sprawi, że zostanie ona zaliczona do tego dżetu \cite{Cacciari:2007fd}
		\item gęstość tła (w danym zdarzeniu)
		\item $N_{SV}$ -- liczba wtórych wierzchołków
		\item $N_{Constit}$ -- liczba cząstek tworzących dżet
	\end{itemize}
	
	
	\item Zmienne opisujące cząstki tworzące dżet:
	\begin{itemize}	
		\item $\eta$, $\phi$ -- pseudopospieszność i kąt azymutalny cząstki względem osi dżetu
		\item $p_T$ -- pęd poprzeczny cząstki
		\item $IP_D$ -- rzut na kierunek poprzeczny wektora parametru zderzenia
		\item $IP_Z$ -- rzut na oś $z$ wektora parametru zderzenia
	\end{itemize}	
	
	
	\item Zmienne opisujące wtórne wierzchołki:
	\begin{itemize}	
		\item $L_{xy}$ -- odległość między pierwotnym a wtórnym wierzchołkiem \angterm{decay length}
		\item $\sigma_{Lxy}$ -- niepewność wyznaczenia $L_{xy}$
		\item $\sigma_{vertex} = \sqrt{d_1^2 + d_2^2 + d_3^2}$ -- rozrzut śladów \angterm{tracks} wokół wtórnego wierzchołka, gdzie $d_i$ to odległość najbliższego zbliżenia śladu do wtórnego wierzchołka (DCA) % https://arxiv.org/pdf/1511.04380.pdf (str. 92)
		\item $M_{inv} = \sqrt{(E_1 + E_2 + E_3)^2 - (\vec{p_1} + \vec{p_2} + \vec{p_3})^2}$ -- masa niezmiennicza wierzchołka, gdzie $E_i, p_i$ to energia i pęd $i$-tej cząstki tworzącej wierzchołek
		\item $\chi^2/Ndf$  dopasowania wtórnego wierzchołka
	\end{itemize}
\end{itemize}


Dżety różnią się liczbą cząstek je tworzących oraz liczbą wtórnych wierzchołków. Większość algorytmów uczenia maszynowego wymaga natomiast dostarczenia danych w postaci tabelarycznej (macierzowej), ze stałą liczbą kolumn (wiersze stanowią kolejne dżety). Aby spełnić to wymaganie konieczne jest przyjęcie pewnej ustalonej liczby wtórnych wierzchołków oraz cząstek tworzących dżet -- w przypadku gdy dżet ma więcej elementów tego typu są one odrzucane, natomiast puste pola są wypełniane zerami w przypadku gdy ma ich mniej. 
Po przeanalizowaniu rozkładów liczby wtórnych wierzchołków i cząstek tworzących dżet (Rys. \ref{fig:nsv_nconstit_distr}) oraz wstępnym sprawdzeniu jak dodawanie kolejnych elementów wpływa na otrzymywane wyniki (na podstawie wzmacnianych drzew decyzyjnych ze względu na wspomnianą w \ref{subsec:dyskusja-2algo} szybkość i stabilność) ustalono liczbę cząstek tworzących dżet równą 15 a wtórnych wierzchołków równą 20.

\begin{figure}[h]
	\centering
	\includegraphics[trim={0.6cm 0cm 0.2cm 0cm},clip,width=0.49\textwidth]{hist-nconstit-cut.png}
	\includegraphics[trim={0.6cm 0cm 0.2cm 0cm},clip,width=0.49\textwidth]{hist-nsv-cut.png}
	\caption{Rozkłady liczby cząstek tworzących dżet i liczby wtórnych wierzchołków (część) wraz wartościami cięć.}
	\label{fig:nsv_nconstit_distr}
\end{figure}

Istotnym zagadnieniem jest także kolejność w jakiej ułożone będą zmienne. 
Dla sieci konwolucyjnych szukających lokalnych zależności rozsądne jest ułożenie obok siebie tych samych zmiennych, np. $L_{xy,1}, L_{xy,2}, L_{xy,3} \dots \sigma_{vertex,1}, \sigma_{vertex,2}, \sigma_{vertex,3} \dots$. 
Dla sieci w pełni połączonych oraz drzew decyzyjnych kolejność zmiennych nie ma znaczenia, ale ważne jest aby ich położenie było stałe, np. aby $L_{xy}$ i $\sigma_{Lxy}$ danego wtórnego wierzchołka były w tych samych miejscach, tak aby możliwe było szukanie zależności między nimi.

Następną kwestią jest wybór wielkości decydującej o kolejności ułożenia elementów, tj. która cząstka będzie cząstką nr 1 a która nr 5. Losowe ułożenie elementów sprawiłoby, że bezpośrednie porównywanie wielkości w danych kolumnach (co ma miejsce bezpośrednio w drzewach decyzyjnych a pośrednio w sieciach neuronowych) straciłoby sens. Z kolei dobry dobór tej kolejności pozwala na łatwe odtworzenie przez algorytm uczenia maszynowego motywowanych fizycznie algorytmów omówionych w sekcji \ref{subsubsec:przeglad-algo}. Przykładowo cięcie na wartość $IP$ drugiej lub trzeciej cząstki (gdy są one posortowane wg malejących wartości $IP$) jest istotą algorytmu nazywanego \textit{Track Counting -- TC} stosowanego w CMS i ALICE.
Kolejność w jakiej ułożone będą elementy, wpływa także na to, które z nich będą odrzucone w przypadku gdy dżet zawiera więcej niż 15 cząstek i 20 wierzchołków. 
Ponownie posiłkowano się testami z użyciem drzew decyzyjnych. Ostatecznie wtórne wierzchołki ułożono według malejącego $L_{xy}$ a cząstki -- malejącego $p_T$.

%Ostatecznie liczba zmiennych użytych do trenowania algorytmów uczenia maszynowego była równa: 