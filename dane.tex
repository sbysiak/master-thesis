\section{Dane}
\label{sec:dane}

%p-p, 13 TeV - Pythia8 - JJ events anchored to LHC16i, ALIROOT-7268

Dane użyte w analizie pochodzą z symulacji Monte Carlo zderzeń proton-proton przy energii w układzie środka masy równej $\sqrt{s} = 13$ TeV dostępnych na serwerach eksperymentu ALICE.
Są to pełne symulacje detektora ALICE, wykorzystujące generator zderzeń \texttt{Pythia8} \cite{Sjostrand:2007gs} (wersja: \textit{Monash2013} \cite{Skands:2014pea}) oraz pakiet \texttt{Geant3} \cite{Brun:1994aa} do transportu cząstek przez materiał detektora. Korzystano ze specjalnych symulacji w tzw. \textit{binach $p_T$}, które zapewniają lepszą statystykę dla wysokich pędów.

Do rekonstrukcji dżetów wykorzystany został algorytm \textit{anti-kt} z parametrem $R = 0.4$ zaimplementowany w pakiecie \texttt{FASTJET} \cite{Cacciari:2011ma}. Dżetów poszukiwano wyłącznie wśród cząstek naładowanych \angterm{charged jets} ze względu na słabe pokrycie przestrzeni fazowej przez kalorymetry w eksperymencie ALICE.
%Symulacje prowadzone były w wąskich binach $p_T$, które zostały połączone.

Do analizy wybrano dżety o $p_T$ większym niż 15 GeV i mieszczące się w całości w akceptancji detektora \textit{TPC}, tj. $|\eta| < 0.9$, co przy użytym parametrze rozmiaru dżetu $R = 0.4$, daje ograniczenie na pseudopospieszność $|\eta| < 0.5$ dla osi dżetu.

\begin{figure}[ht]
	\centering
	\includegraphics[width=0.5\textwidth]{sv-ip-def.png}
	\caption{Rysunek ilustrujący znaczenie używanych wielkości: $L_{xy}$ oraz $IP$. \imgsrc{\cite{sv-def-d0exp}}.}
	\label{fig:sv-ip-def}
\end{figure}

Dla każdego dżetu obliczony został szereg wielkości, które można podzielić na zmienne charakteryzujące dżet, opisujące wtórne wierzchołki lub cząstki tworzące dżet. 
Za potencjalne wtórne wierzchołki uznaje się kombinacje trzech cząstek spełniających pewne dosyć luźne kryteria jak $p_T > 0.15$ GeV (rozważane są wyłącznie trzy-cząstkowe wtórne wierzchołki), stąd ich liczba może być dużo większa od liczby cząstek tworzących dżet.



Lista używanych zmiennych:
\begin{itemize}
	\item Zmienne charakteryzujące dżet:
	\begin{itemize}	
		\item $\eta_{jet}$, $\phi_{jet}$ -- pseudopospieszność i kąt azymutalny osi dżetu
		\item $p_{T,\;jet}$ -- pęd poprzeczny dżetu
		\item $M_{jet} = \sqrt{ \left(\sum_i E_i\right)^2  -  \left(\sum_i \vec{p_i}\right)^2 }$ -- masa dżetu, gdzie $E_i$ i $p_i$ to energia i pęd kolejnych cząstek tworzących dżet
		\item $A_{jet}$ -- powierzchnia dżetu wyliczana przez algorytm \textit{kt} w płaszczyźnie ($\eta$, $\phi$). Do powierzchni dżetu zaliczany jest każdy element powierzchni, w którym dodanie cząstki o nieskończenie małym pędzie poprzecznym sprawi, że zostanie ona zaliczona do tego dżetu \cite{Cacciari:2007fd}
		\item $\rho_{\;bckg}$ -- gęstość tła w danym zdarzeniu
	\end{itemize}
	
	\item Zmienne opisujące cząstki tworzące dżet (składniki dżetu):
	\begin{itemize}	
		\item $\eta$, $\phi$ -- pseudopospieszność i kąt azymutalny cząstki względem osi dżetu
		\item $p_T$ -- pęd poprzeczny cząstki
		\item $IP_D$ -- rzut na kierunek poprzeczny wektora $IP$, tj. odległości najbliższego zbliżenia
		\item $IP_Z$ -- rzut na oś $z$ wektora $IP$
		\item $N_{Constit}$ -- liczba cząstek tworzących dżet
	\end{itemize}	
	
	
	\item Zmienne opisujące wtórne wierzchołki:
	\begin{itemize}	
		\item $L_{xy}$ -- odległość między pierwotnym a wtórnym wierzchołkiem \angterm{decay length} w płaszczyźnie $x-y$
		\item $\sigma_{Lxy}$ -- niepewność wyznaczenia $L_{xy}$
		\item $\sigma_{vertex} = \sqrt{d_1^2 + d_2^2 + d_3^2}$ -- rozrzut śladów \angterm{tracks} wokół wtórnego wierzchołka, gdzie $d_i$ to to najmniejsza odległość pomiędzy śladem \textit{i}-tej cząstki a wtórnym wierzchołkiem
		%gdzie $d_i$ to odległość najbliższego zbliżenia śladu do wtórnego wierzchołka (DCA) % https://arxiv.org/pdf/1511.04380.pdf (str. 92)
		\item $M_{inv} = \sqrt{(E_1 + E_2 + E_3)^2 - (\vec{p_1} + \vec{p_2} + \vec{p_3})^2}$ -- masa niezmiennicza wierzchołka, gdzie $E_i,\;\vec{p_i}$ to energia i pęd $i$-tej cząstki tworzącej wierzchołek
		\item $\chi^2/Ndf$ -- jakość dopasowania wtórnego wierzchołka
		\item $N_{SV}$ -- liczba wtórych wierzchołków

	\end{itemize}
\end{itemize}

Dżety różnią się liczbą cząstek je tworzących oraz liczbą wtórnych wierzchołków. Większość algorytmów uczenia maszynowego wymaga natomiast dostarczenia danych w postaci tabelarycznej (macierzowej), ze stałą liczbą kolumn (wiersze stanowią kolejne dżety). Aby spełnić to wymaganie konieczne jest przyjęcie pewnej ustalonej liczby wtórnych wierzchołków oraz cząstek tworzących dżet -- w przypadku gdy dżet ma więcej elementów tego typu są one odrzucane, natomiast puste pola są wypełniane zerami w przypadku gdy ma ich mniej. 
Po przeanalizowaniu rozkładów liczby wtórnych wierzchołków i cząstek tworzących dżet (Rys. \ref{fig:nsv_nconstit_distr}) oraz wstępnym sprawdzeniu jak dodawanie kolejnych elementów wpływa na otrzymywane wyniki (na podstawie wzmacnianych drzew decyzyjnych ze względu na wspomnianą w \ref{subsec:dyskusja-2algo} szybkość i stabilność) przyjęto liczbę cząstek tworzących dżet równą 15 a liczbę wtórnych wierzchołków równą 20.

\begin{figure}[ht]
	\centering
	\includegraphics[trim={0.6cm 0cm 0.2cm 0cm},clip,width=0.49\textwidth]{hist-nconstit-cut.png}
	\includegraphics[trim={0.6cm 0cm 0.2cm 0cm},clip,width=0.49\textwidth]{hist-nsv-cut.png}
	\caption{Rozkłady liczby cząstek tworzących dżet i liczby wtórnych wierzchołków (pokazana jest tylko część, rozkład sięga $N_{SV} = 200$) wraz wartościami cięć.}
	\label{fig:nsv_nconstit_distr}
\end{figure}

Istotnym zagadnieniem jest także kolejność w jakiej ułożone będą zmienne. 
Dla sieci konwolucyjnych szukających lokalnych zależności dobrze jest pogrupować zmienne, tak aby obok siebie znajdowały się te same wielkości fizyczne, np. $L_{xy,1}, L_{xy,2}, L_{xy,3} \dots \sigma_{vertex,1}, \sigma_{vertex,2}, \sigma_{vertex,3} \dots$ -- tak też zrobiono.
Dla sieci w pełni połączonych oraz drzew decyzyjnych kolejność zmiennych nie ma znaczenia, ale ważne jest aby położenie zmiennych było ustalone dla kolejnych wierzchołków lub cząstek tzn. żeby $L_{xy}$ i $\sigma_{Lxy}$ danego wtórnego wierzchołka były w tych samych miejscach, tak aby możliwe było szukanie zależności między nimi.

Następną kwestią jest wybór wielkości decydującej o kolejności ułożenia elementów, tj. która cząstka będzie cząstką nr 1 a która nr 5. Losowe ułożenie elementów sprawiłoby, że bezpośrednie porównywanie wartości w danej kolumnach (co ma miejsce bezpośrednio w drzewach decyzyjnych a pośrednio w sieciach neuronowych) straciłoby sens. Z kolei dobry dobór tej kolejności pozwala na łatwe odtworzenie przez algorytm uczenia maszynowego motywowanych fizycznie algorytmów omówionych w sekcji \ref{subsubsec:przeglad-algo}. Przykładowo cięcie na wartość $IP$ drugiej lub trzeciej cząstki (gdy są one posortowane wg malejących wartości $IP$) jest istotą algorytmu nazywanego \textit{Track Counting -- TC} stosowanego w CMS i ALICE.
Kolejność w jakiej ułożone będą elementy, wpływa także na to, które z nich będą odrzucone w przypadku gdy dżet zawiera więcej niż 15 cząstek lub 20 wierzchołków. 
Ponownie posiłkowano się testami z użyciem drzew decyzyjnych. Ostatecznie wtórne wierzchołki ułożono według malejącego $L_{xy}$ a cząstki -- malejącego $p_T$.

Zbiór danych użytych w analizie liczył 46 000 dżetów, z czego ponad 20 tysięcy stanowiły dżety \textit{b} (trening przeprowadzano przy zrównoważonej liczbie przykładów dla sygnału i tła).
Z tego zbioru zbudowano trzy zestawy danych (każdy zawierał wszystkie 46 tys. wierszy, różniły się kolumnami): \textit{SV}, który zawierał tylko zmienne związane z wtórnymi wierzchołkami, \textit{constit} -- zmienne opisujące cząstki tworzące dżet \angterm{constituents} oraz \textit{merged} -- wszystkie zmienne.
%Ostatecznie liczba zmiennych użytych do trenowania algorytmów uczenia maszynowego była równa: 