\begin{appendices}
  \renewcommand\thetable{\thesection\arabic{table}}
  \renewcommand\thefigure{\thesection\arabic{figure}}
  \setcounter{table}{0}

\section{Metryki}
\label{sec:app:metryki}
W poniżej tabeli zebrano stosowane najczęściej miary jakości klasyfikatorów.
We wzorach definiujących metryki wykorzystano następujące wielkości:\\
TP \angterm{true positives} -- liczba poprawnie zaklasyfikowanych przypadków klasy pozytywnej\\
TN \angterm{true negatives} -- liczba poprawnie zaklasyfikowanych przypadków klasy negatywnej\\
FP \angterm{false positives} -- liczba błednie zaklasyfikowanych przypadków klasy negatywnej\\
FN \angterm{false negatives} -- liczba błędnie zaklasyfikowanych przypadków klasy pozytywnej\\

%\setlength{\extrarowheight}{20pt}

\begin{table}[ht]
\centering
\begin{tabular}{llc}
\toprule
nazwa metryki		& 	nazwa angielska				& 	wzór \\ 
\bottomrule
\toprule
dokładność			&	accuracy					& (TP+TN) / (TP+TN+FP+FN) \\ 
\midrule
precyzja			&	precision					& TP / (TP+FP) \\ 
\midrule
\pbox{0.3\textwidth}{czułość,\\ wydajność id. sygnału}				& 	\pbox{0.3\textwidth}{recall,\\ sensitivity,\\ TP Rate}	& TP / (TP+FN) \\ 
\midrule
swoistość			& 	\pbox{0.3\textwidth}{specificity,\\ TN Rate}			& TN / (TN+FP) \\ 
\midrule
F1					& 	F1							& 2 $\frac{precision\;\cdot\;recall}{precision\;+\;recall}$ \\ 
\midrule
prawd. błędnej klas. tła & \pbox{0.3\textwidth}{mistagging rate,\\ FP Rate}		& FP / (FP+TN) \\
\bottomrule
\end{tabular}
\caption{Tabela zawierająca nazwy i definicje popularnych metryk.}
\label{tab:metryki}
\end{table}

Podanie tylko jednej metryki jest zwykle niewystarczające i stosuje się pary metryk, np. precyzja - czułość. Jeśli predykcja klasyfikatora ma charakter ciągły, to poprzez zmienianie wartości progowej otrzymuje się różne punkty pracy, scharakteryzowane przez wartości obu metryk. Kolejne punkty pracy wykreślone np. na wykresie $TPR(FPR)$ dają krzywą nazywaną krzywą ROC. 
Wykres ten w przypadku klasyfikatora losowego jest prostą łączącą punkty o współrzędnych (0,0) i (1,1). Im lepszy klasyfikator tym bardziej wykres wygięty jest w stronę punktu (0,1).
Pole pod tą krzywą \angterm{ROC Area Under Curve -- ROC AUC} jest kolejną metryką, bardzo często wykorzystywaną w praktyce, gdyż łączy w sobie wszystkie możliwe punkty pracy, jest także odporne na niezrównoważenie klas. Pole pod krzywą \textit{ROC} przyjmuje wartość 0.5 dla klasyfikatora losowego oraz 1 dla idealnego.

W literaturze dot. klasyfikacji dżetów wyniki zwykle przedstawia się z użyciem dwóch powszechnie znanych metryk, ale o zmienionych nazwach:
\textit{True Positive Rate}, nazywanej \textit{b jet tagging efficiency} -- wydajnością na identyfikację dżetów \textit{b} oraz \textit{False Positive Rate}, nazywanej \textit{mistagging rate}  -- prawdopodobieństwem błędnej klasyfikacji dżetów tła jako dżety \textit{b}.

\clearpage
\FloatBarrier
\section{Skróty i oznaczenia}
\label{sec:app:skroty}
W pracy używane są następujące skróty i oznaczenia: 
\begin{itemize}
	\item algorytmy: \\ \textit{FC} -- sieci neuronowe w pełni połączone,\\ \textit{Conv} - sieci neuronowe konwolucyjne,\\ \textit{BDT} - wzmacniane drzewa decyzyjne
	\item zbiory danych: \\ \textit{SV} -- zestaw zawierający tylko zmienne związane z wtórnymi wierzchołkami,\\ \textit{constit} -- zestaw zawierający tylko zmienne związane z cząstkami tworzącymi dżet,\\ \textit{merged} -- zestaw zawierający wszystkie zmienne (tj. \textit{SV}+\textit{Constit}+zmienne charakteryzujące dżet jako całość)
	\item modele (zestaw danych + algorytm) nazywane są wg wzoru: \\ (oznaczenie\_zbioru\_danych)-(oznaczenie\_algorytmu), \\np. \textit{SV-Conv} oznacza konwolucyjną sieć neuronową wytrenowaną na zmiennych związanych z wtórnymi wierzchołkami a \textit{merged-BDT} - wzmacniane drzewa decyzyjne, do treningu których użyte zostały wszystkie zmienne. Zapis \textit{merged-X} stanowi zbiorcze oznaczenie modeli \textit{merged-SV}, \textit{merged-Conv} i \textit{merged-BDT}.
	\item poszczególne zmienne (kolumny w zbiorze danych) oznaczane są według wzoru: \\ (nazwa\_zmiennej) -- (numer\_obiektu), \\np. $\sigma_{Lxy} - SV2$ oznacza niepewność wyznaczenia $L_{xy}$ dla wtórnego wierzchołka nr 2, a $IP_Z\,-\,C5$ -- rzut odległości najbliższego zbliżenia cząstki nr 5 na oś wiązki -- numery na posortowanych listach (Rozdz. \ref{sec:dane}, gdzie znajdują się także opisy wielkości fizycznych).
\end{itemize} 


\end{appendices}
