\clearpage
\section{Uczenie maszynowe}
\label{sec:ml}

Uczenie maszynowe jest bardzo szerokim i obecnie dynamicznie się rozwijającym obszarem wiedzy. Występuje w wielu odmianach łącząc w sobie w zależności od wariantu wiele dziedzin takich jak matematyka (statystyka, algebra) informatyka (algorytmika, teoria informacji) a także elementy robotyki i sterowania. 
Dziedzinami, w których jest najczęściej wykorzystywane są min. widzenie maszynowe, przetwarzanie języka naturalnego, autonomiczne roboty i pojazdy, systemy decyzyjno - eksperckie, optymalizacyjne oraz rekomendacyjne.

W tej pracy wykorzystywana jest gałąź uczenia maszynowego nazywana uczeniem nadzorowanym lub "uczeniem z nauczycielem" \angterm{supervised learning}, gdzie uczenie występuje na podstawie poprawnie oznaczonych przykładów. Terminami bliskoznacznymi dla tak rozumianego uczenia maszynowego są uczenie statystyczne \angterm{statistical learning} i rozpoznawanie wzorców \angterm{patterm recognition}.

Problem identyfikacji dżetów jest klasycznym przykładem zagadnienia klasyfikacji, gdzie poprawna odpowiedź jest jedną ze skończonej ilości opcji (klas) w przeciwieństwie do regresji, gdzie szukana odpowiedź algorytmu ma charakter ciągły. 

Występuje wiele algorytmów uczenia maszynowego takich jak regresja liniowa i logistyczna, drzewa decyzyjne i ich wariacje, maszyny wektorów wspierających, sztuczne sieci neuronowe oraz wiele innych. Uczenie polega na znalezieniu pewnej funkcji dopasowującej do przyjmowanego na wejściu zestawu (wektora) cech (zmiennych, kolumn) pewną odpowiedź (predykcję), która minimalizuje zadaną funkcję straty. Jej rolę w przypadku regresji często pełni błąd średniokwadratowy a w przypadku klasyfikacji np. entropia krzyżowa \angterm{cross entropy} [\#PRZYPIS]. 
Różne algorytmy szukają przy tym funkcji dopasowującej należącej do różnych klas funkcji: przykładowo klasyczne drzewa decyzyjne przeszukują tylko przestrzeń funkcji dających się opisać skończonym zbiorem reguł "jeśli -- to" \angterm{if -- else}.

W pracy wykorzystane zostały dwa rodzaje algorytmów: wzmacniane drzewa decyzyjne oraz sieci neuronowe.

\subsection{Wzmacniane drzewa decyzyjne}
\label{subsec:drzewa}

Wzmacniane drzewa decyzyjne są jednym z rozwinięć klasycznego algorytmu drzewa decyzyjnego. 
Pojedyncze drzewo decyzyjne dzieli przestrzeń cech uczących przy pomocy prostopadłych cięć, na mniejsze/większe niż zadana wartość w przypadku zmiennej ciągłej lub na należące/nie należące do danej klasy w przypadku zmiennej kategorycznej. Każdy podział, nazywany węzłem, daje dwie gałęzie, które można dalej niezależnie dzielić aż do ostatniego poziomu (liści). Kolejne podziały wybierane są tak, aby zbiory przykładów wpadające do poszczególnych gałęzi były jak najbardziej jednorodne. Stosuje się różne miary jednorodności takie tak: \textit{indeks Gini}

Drzewa decyzyjne są często łączone w komitety klasyfikatorów \angterm{ensemble methods}. Wiele "słabych" klasyfikatorów jest łączonych w jeden "silny" na dwa sposoby: \textit{bagging} oraz \textit{boosting} (wzmacnianie), które są często ze sobą porównywane.

\textit{Bagging} - w zastosowaniu dla drzew decyzyjnych nazywany algorytmem lasów losowych \angterm{random forest} - polega na wytrenowaniu wielu drzew, każde na podstawie $N$ przykładów losowo wylosowanych z powtórzeniami spośród $N$-licznego zbioru treningowego. Dodatkowo, do uczenia każdego drzewa używa się tylko podzbioru wszystkich cech uczących. Końcową predykcję algorytmu otrzymuje się poprzez "głosowanie" wszystkich drzew z odpowiednimi wagami.

\textit{Boosting} 
ToDo: Boosting - AdaBoost, porównanie baggVSboost - rownoległyVSsekwencyjny

\subsection{Sieci neuronowe}
\label{subsec:nn}

Sieci neuronowe \angterm{neural networks -- NN} są szczególnym algorytmem uczenia maszynowego. Występują w bardzo wielu odmianach i są wykorzystywane w rozwiązywaniu szerokiej gamy problemów. Nawet bardzo pobieżny opis sieci neuronowych wymaga dużo więcej miejsca niż może być temu poświęcone w tej pracy. 
Wprowadzenia do sieci neuronowych od podstaw można znaleźć m.in. w \cite{samarasinghe2016neural} lub \cite{Goodfellow-et-al-2016}. 
Tu przedstawione zostaną wyłącznie wybrane zagadnienia mające ściślejszy związek z pracą. Używane mogą być terminy, których znaczenie wyjaśniane jest w podanych źródłach.

W niniejszej pracy, wykorzystane zostały dwa rodzaje sieci neuronowych: sieci w pełni połączone \angterm{fully connected NN -- FC NN}, nazywane także wielowarstwowymi perceptronami \angterm{multi-layer perceptron -- MLP} oraz sieci konwolucyjne \angterm{convolutional NN -- ConvNets, CNN}. 

\subsubsection*{Sieci w pełni połączone}
W nierekurencyjnych sieciach neuronowych (tylko takie są używane w tej pracy), informacja jest przekazywana kolejno od warstw wejściowych, poprzez warstwy ukryte aż do wyjściowej. 
W sieciach typu \textit{FC} wszystkie warstwy składają się z identycznych neuronów -- każdy neuron dostaje na wejściu wektor, natomiast zwraca skalar -- wartość pewnej zadanej, nieliniowej funkcji, jako argument podając średnią ważoną z elementów wektora wejściowego. Wartości zwracane przez neurony w danej warstwie składają się na wektor wejściowy dla neuronów kolejnej warstwy. Wejściem dla pierwszej warstwy są natomiast kolejne przykłady ze zbioru uczącego. Trenowanie sieci neuronowych polega na zmienianiu wag (parametrów) w liczonej w każdym neuronie średniej, każdy neuron posiada własny, niezależny zestaw wag.

Istnieje twierdzenie o sieciach neuronowych jako uniwersalnych aproksymatorach funkcji \angterm{universal approximation theorem} \cite{hornik91}, mówiące, że już sieć neuronowa o jednej warstwie ukrytej jest zdolna do przybliżenia dowolnej funkcji z dowolną dokładnością. Twierdzenie to nie podaje niestety liczby potrzebnych neuronów a przede wszystkim -- sposobu ich trenowania. 
Trenowanie jest prostsze w przypadku zastosowania wielu warstw, które odpowiadają kolejnym poziomą abstrakcji jednak nadal jest dużym wyzwaniem ze względu na fakt, że nawet stosunkowo niewielka sieć może posiadać bardzo dużą liczbę parametrów, przykładowo sieć o czterech warstwach, w każdej po 128 neuronów ma ich ponad 65 tysięcy.


\subsubsection*{Sieci konwolucyjne}
Jednym ze sposobów na ograniczenie liczby trenowanych parametrów jest użycie konwolucyjnych sieci neuronowych \cite{krizhevsky2012imagenet} (bardziej poprawną choć rzadko używaną nazwą w języku polskim jest sieć splotowa). Są one inspirowane połączeniami w korze wzrokowej zwierząt i wywodzą się z badań w obszarze widzenia komputerowego, gdzie liczby parametrów są szczególnie duże (wektor wejściowy ma wymiar równy liczbie pikseli w obrazie), na takim przykładzie również najłatwiej zrozumieć ich działanie.

Sieci konwolucyjne różnią się od sieci typu \textit{FC} tym, że część wag połączeń między warstwami jest dzielona. Występuje w nich nowy rodzaj warstwy, nazywany warstwą konwolucyjną. Każda jednostka w warstwie konwolucyjnej (filtr) ma pewną stałą (niewielką) liczbę wag. Połączenie z dużym  wejściem realizowane jest przez powielanie tych samych wag w połączeniach z kolejnymi fragmentami wektora wejściowego (Rys. \ref{fig:convolution}). Rezultatem działania filtra na macierz jest wynik operacji splotu. Liczba parametrów przypadająca na każdy filtr jest równa jego rozmiarowi i nie zależy od wielkości wektora wejściowego. 

\begin{figure}[ht]
	\centering
	\includegraphics[width=0.6\textwidth]{convolution_matrix.png}
	\caption{Schemat działania pojedynczego filtra z warstwy konwolucyjnej (operacja splotu). \imgsrc{\cite{Velic17asd}}.}
	\label{fig:convolution}
\end{figure}

W przypadku gdy zamiast wejścia dwuwymiarowego (jak np. obraz czarno-biały), mamy do czynienia z wejściem trójwymiarowym (np. trzeci wymiar to kolejne kolory w kodowaniu RGB), filtry również muszą mieć trzy wymiary, przy czym rozmiar w ostatnim wymiarze musi być równy rozmiarowi w tym kierunku wektora wejściowego. Wynik operacji splotu jest ponownie dwuwymiarowy, gdyż filtr przesuwany jest tylko w dwóch pierwszych wymiarach. Trzeci wymiar powstaje przez składanie kolejnych filtrów. Widać zatem, że również w przypadku gdy na wejścia podawany jest obraz czarno-biały, filtry w kolejnych warstwach konwolucyjnych (oprócz pierwszej) mają po trzy wymiary.

\begin{figure}[ht]
	\centering
	\begin{minipage}[c]{0.45\textwidth}
		\includegraphics[width=\textwidth]{conv-over-volume.png}
		\caption{Działanie pojedynczego filtra~(3D) na wejście o trzech wymiarach. \imgsrc{{\cite{Ng17ert}}}.}
		\label{fig:conv-over-vol}
	\end{minipage}
	\hspace{3em}
	\begin{minipage}[c]{0.45\textwidth}
		\vspace{-0.5em}
		\includegraphics[width=0.9\textwidth]{max-pool.png}
		\vspace{1.5em}
		\caption{Działanie warstwy typu \\ \textit{max-pool}. \imgsrc{\cite{Velic17asd}}.}
		\label{fig:max-pool}
	\end{minipage}	
\end{figure}




%\begin{wrapfigure}[11]{r}{0.5\textwidth}
%  \begin{center}
%  	\vspace{-2em}
%	\includegraphics[width=0.45\textwidth]{conv-over-volume.png}
%	\begin{minipage}[c]{0.48\textwidth}
%	\vspace{1em}
%    \caption{Działanie pojedynczego filtra~(3D) na wejście o trzech wymiarach.}
%   	\label{fig:conv-over-vol}
%    \end{minipage}
%
%  \end{center}
%\end{wrapfigure}
%
%W przypadku gdy zamiast wejścia dwuwymiarowego (jak np. obraz czarno-biały), mamy do czynienia z wejściem trójwymiarowym (np. trzeci wymiar to kolejne kolory w kodowaniu RGB), filtry również muszą mieć trzy wymiary, przy czym rozmiar w ostatnim wymiarze musi być równy rozmiarowi w tym kierunku wektora wejściowego. Wynik operacji splotu jest ponownie dwuwymiarowy, gdyż filtr przesuwany jest tylko w dwóch pierwszych wymiarach. Trzeci wymiar powstaje przez składanie kolejnych filtrów. Widać zatem, że również w przypadku gdy na wejścia podawany jest obraz czarno-biały, filtry w kolejnych warstwach konwolucyjnych (oprócz pierwszej) mają po trzy wymiary.
%
%
%\begin{wrapfigure}[12]{r}{0.5\textwidth}
%  \begin{center}
%  	\vspace{-2em}
%	\includegraphics[width=0.45\textwidth]{max-pool.png}
%	\begin{minipage}[c]{0.48\textwidth}
%	\vspace{1em}
%    \caption{Działanie warstwy typu \textit{max-pool}.}
%   	\label{fig:conv-over-vol}
%    \end{minipage}
%
%  \end{center}
%\end{wrapfigure}

Oprócz warstw konwolucyjnych, w sieciach tego typu stosowane są także tzw. warstwy typu \textit{max-pooling}. Zasada jej działania jest bardzo prosta: wykonuje funkcję \textit{maksimum} na zadanym fragmencie obrazu (Rys. \ref{fig:max-pool}). Ich rolą jest zmniejszanie rozmiaru przekazywanej w sieci informacji.

Typowa architektura stosowana w przypadku sieci konwolucyjnych jest następująca: najpierw warstwy konwolucyjne (pomiędzy nimi czasem warstwy typu \textit{max-pool}), następnie wszystkie filtry są rozwijane i składane w długi jednowymiarowy wektor, który przekazywany jest do warstw typu \textit{FC}. W przypadku problemu klasyfikacji, na końcu znajduje się jeszcze warstwa typu \textit{softmax} normalizująca wyjście z sieci do jedynki (Rys. \ref{fig:cnn-arch}). 

\begin{wrapfigure}[10]{r}{0.3\textwidth}
  \begin{center}
  	\vspace{-1em}
	\includegraphics[width=0.2\textwidth]{translation-inv-non-face.png}
	\caption{\imgsrc{\cite{tan17lkj}}.}
   	\label{fig:translation-inv-nonface}
  \end{center}
\end{wrapfigure}


\begin{figure}[ht]
	\centering
	\includegraphics[width=\textwidth]{cnn-typical-architecture.jpeg}
	\caption{Typowa struktura stosowana w sieciach konwolucyjnych. Warstwy konwolucyjne mają za zadanie wydobywać cechy, na podstawie których późniejsze warstwy dokonują klasyfikacji.
	Widoczna jest charakterystyczna stopniowa zmiana rozmiaru przekazywanej macierzy: rozmiar poprzeczny maleje kosztem głębokości, co odpowiada rosnącej liczbie filtrów i malejącemu rozmiarowi obszaru po jakim są one przesuwane. \imgsrc{\cite{mathworksXXpos}}.}
	\label{fig:cnn-arch}
\end{figure}



Sieci konwolucyjne posiadają dwie właściwości odróżniające je od \textit{MLP}:
\begin{itemize}
	\item niezmienniczość względem przesunięcia \angterm{translation invariance} -- głównie za sprawą dzielenia wag oraz obecności warstw typu \textit{max-pool}, położenie danej cechy na obrazie jest niemal bez znaczenia (obraz po prawej stronie byłby rozpoznany jako twarz)
	\item lokalność połączeń -- filtry obejmują tylko kilka sąsiednich pikseli (tam zwykle występują najsilniejsze zależności) nie są w stanie dostrzec cechy rozciągniętej na obszar większy od rozmiaru filtra
\end{itemize}

\subsubsection*{Hiperparametry i trenowanie sieci neuronowych na analizowanych danych}

Parametry sieci, których wartości są określane przez projektanta sieci, takie jak liczba warstw ukrytych są nazywane hiperparametrami (dla odróżnienia od parametrów - wag połączeń).

Testowane były trzy architektury: sieci w pełni połączone, sieci konwolucyjne oraz sieć złożona z dwóch gałęzi, osobnych dla wtórnych wierzchołków i cząstek tworzących dżet (Rozdz. \ref{sec:dane}) przedstawione schematycznie na Rys. \ref{fig:nets-params}.
Do trenowania sieci wykorzystano wysokopoziomową bibliotekę \texttt{Keras} \cite{chollet15} korzystającą z silnika obliczeniowego zaimplementowanego w \texttt{TensorFlow} \cite{tensorflow15}.

\begin{figure}[ht]
	\centering
	\includegraphics[width=\textwidth]{networks_params4.png}
	\caption{Schematyczne przedstawienie trzech testowanych rodzin architektur sieci. Każdy blok odpowiada kilku warstwom danego typu. Bloki warstw typu \textit{FC}  i opisane jako "kompresja" składały się z kilku neuronów i miały za zadanie zredukować całą informację z danej gałęzi do wektora kilku liczb.}
	\label{fig:nets-params}
\end{figure}

\clearpage
Zestaw hiperparametrów definiujący działanie sieci w pełni połączonej: 
\begin{itemize}
	\item liczba warstw ukrytych 
	\item liczba neuronów w każdej warstwie
	\item funkcja aktywacji -- nieliniowa funkcja aplikowana przed zwróceniem wartości w każdym neuronie, najpopularniejsze to \textit{tanh}, \textit{ReLU} ($f(x) = max(0,x)$) oraz funkcja sigmoidalna ($f(x) = \frac{1}{1+exp(x)}$)
	\item algorytm optymalizacyjny -- spadek gradientowy lub jego wariacje
	\item parametr szybkości uczenia i jego modyfikacje w trakcie uczenia
	\item liczba przykładów trenujących przetwarzanych w jednym kroku uczenia \angterm{batch\_size} -- im większy tym szybsze jest trenowanie sieci (dzięki wydajnym operacjom macierzowym), natomiast może się to odbywać kosztem precyzji
	\item liczba epok uczenia -- ile razy będzie pokazywany sieci każdy przykład
	\item opcjonalnie: warunki stopu \angterm{early stopping}, czyli przerwanie procesu uczenia mające na celu uniknięcie przetrenowania sieci, w momencie gdy błąd popełniany na zbiorze walidacyjnym przestaje maleć
	\item opcjonalnie: regularyzacja przy pomocy różnych technik (zwykle konieczna)
\end{itemize}

Ponadto dla sieci konwolucyjnych: 
	\begin{itemize}
		\item liczba warstw konwolucyjnych i liczba filtrów w każdej warstwie
		\item obecność lub brak warstw \textit{max-pool} i rozmiar ich okna
		\item rozmiar filtrów i długość skoku przy ich przesuwaniu
	\end{itemize}

Same dwie pierwsze wielkości dają nieograniczoną liczbę konfiguracji. 
Czas trenowania sieci neuronowych jest rzędy wielkości większy niż drzew decyzyjnych, dlatego przyjęto szereg kroków mających na celu zmniejszenie przeszukiwanej przestrzeni hiperparametrów. Na podstawie wstępnych testów oraz różnych wskazówek dostępnych w literaturze przyjęto:
\begin{itemize}
	\item \textit{batch\_size} zawsze równy 64 (inne testowane wartości: 16, 32, 128)
	\item za algorytm optymalizacyjny przyjęto algorytm o nazwie \textit{Nadam} \cite{Nadam}, tj. rozwinięcie algorytmu \textit{Adam} \cite{KingmaB14} o parametr Nesterova (inne testowane to zwykły spadek gradientowy oraz \textit{Adam})
	\item funkcję aktywacji: \textit{ReLU}
	\item liczba epok równa 50, 100 lub 200, zrezygnowano z \textit{early stopping}
	\item stałe w trakcie treningu wartości parametru szybkości uczenia 
	\item spośród technik regularyzacyjnych testowano wyłącznie \textit{dropout} \cite{srivastava14} z prawdopodobieństwem odrzucenia równym 0.1, 0.2 lub 0.5
	\item kilka wybranych kombinacji dla zestawu parametrów: rozmiar filtra, długość skoku i rozmiar okna w warstwach \textit{max-pool} -- takie same w kolejnych warstwach
	\item liczby neuronów/filtrów w warstwach będące zawsze potęgami dwójki oraz stałą liczbę w kolejnych warstwach lub zmieniającą się o stały czynnik, np. 256-128-64, 128-128-128 lub 16-32-64
	\item liczba warstw \textit{FC}: 2-8, konwolucyjnych 2-6
\end{itemize}

Nawet po przyjęciu powyższych uproszczeń nie sposób sprawdzić wszystkich możliwych zestawów hiperparametrów, dlatego sposób ich dobierania w kolejnych testach był mocno empiryczny. 
Dostępne dane dzielone były na trzy zbiory: trenujący, walidacyjny i testowy. Wobec braku warunków stopu, zbiór walidacyjny użyty był wyłącznie do porównywania różnych zestawów parametrów, tak aby wynik testowy pozostał nieobciążony.

Zgodnie z zasadą ortogonalizacji działań, proces doboru hiperparametrów dzielono na dwie części: najpierw starano się uzyskać jak najlepsze wyniki na zbiorze uczącym, a dopiero później zmusić algorytm do lepszej generalizacji na zbiorze testowym przez zwiększoną regularyzację i modyfikację parametru szybkości uczenia. 

%Należy jednak pamiętać, że przedstawione w dalszej części pracy wyniki uzyskane dla sieci neuronowych są jedynie rezultatem skończonej ilości prób i nie należy ich traktować jako ostatecznych.






%
%opis hiperparametrów (na podstawie obrazka?)
%
%Conv - opis na 2d, zwykle obrazki, tu: 1D np. szeregi czasowe
%http://www.deeplearningbook.org/contents/convnets.html - str. 329-334

%różnice - lokalne połączenia, dzielenie wag, mniejsza liczba wag w Conv

\subsection{Dyskusja użycia dwóch algorytmów}
Użycie więcej niż jednego algorytmu ma wiele zalet. 
Po pierwsze daje możliwość porównania wyników. Pozwala to na pewne oszacowanie błędu \textit{Bayesowskiego} (najniższego możliwego do osiągnięcia przez jakikolwiek algorytm błędu). Jest to bardzo ważne w sytuacji, gdy nie dysponuje się innym oszacowaniem tego błędu (w wielu problemach naturalnych dla człowieka jak rozpoznawanie obiektów na obrazkach jest nim błąd ludzki lub też błąd popełniany przez zespół ekspertów w bardziej zaawansowanych zastosowaniach).

Po drugie, wykorzystane zostały dwa algorytmy mocno różniące się w swojej naturze, co pozwala wykorzystać cechy każdego z nich w analizie: przykładowo sieci neuronowe dobrze radzą sobie z nieustrukturyzowanymi danymi -- potrafią tworzyć wysoko poziomowe cechy na podstawie niskopoziomowego wejścia (np. położenia oka na zdjęciu twarzy na podstawie pixeli). Są natomiast trudne w interpretacji i często traktowane są jako tzw. "czarne skrzynki" \angterm{black box}. Oprócz tego, liczba możliwych konfiguracji sieci jest ogromna i przez to niemożliwe jest stwierdzenie czy wykorzystane zostały pełne ich możliwości.

Z kolei drzewa decyzyjne posiadają stosunkowo niewielką liczbę parametrów, co pozwala na ich ekstensywne przeszukiwanie i otrzymanie wyników, które można uznać za optymalne dla tego algorytmu.
Ponadto, w przypadku drzew istnieją niewymagające dodatkowych obliczeń miary użyteczności poszczególnych zmiennych, co daje wgląd w działanie algorytmu i poprawia intuicyjne zrozumienie jego predykcji.

\vspace{5em}

drzewa - xgb, nn - keras?

drzewa - opis algorytmy wyjasniajace, wplyw zmiennych

partial dependence - chyba mogą być też dla NN - sprawdzić czy liczy się w skonczonym czasie



procedura-podzial na train/val/test
parameters tuning, CV




