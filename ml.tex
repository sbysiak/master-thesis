\clearpage
\section{Uczenie maszynowe}
\label{sec:ml}

Uczenie maszynowe jest bardzo szerokim i obecnie dynamicznie się rozwijającym obszarem wiedzy. Występuje w wielu odmianach łącząc w sobie w zależności od wariantu wiele dziedzin takich jak matematyka (statystyka, algebra) informatyka (algorytmika, teoria informacji) a także elementy robotyki i sterowania. 
Dziedzinami, w których jest najczęściej wykorzystywane są min. widzenie maszynowe, przetwarzanie języka naturalnego, autonomiczne roboty i pojazdy, systemy decyzyjno - eksperckie, optymalizacyjne oraz rekomendacyjne.

W tej pracy wykorzystywana jest gałąź uczenia maszynowego nazywana uczeniem nadzorowanym lub "uczeniem z nauczycielem" \angterm{supervised learning}, gdzie uczenie występuje na podstawie poprawnie oznaczonych przykładów. Terminami bliskoznacznymi dla tak rozumianego uczenia maszynowego są uczenie statystyczne \angterm{statistical learning} i rozpoznawanie wzorców \angterm{patterm recognition}.

Problem identyfikacji dżetów jest klasycznym przykładem zagadnienia klasyfikacji, gdzie poprawna odpowiedź jest jedną ze skończonej ilości opcji (klas) w przeciwieństwie do regresji, gdzie szukana odpowiedź algorytmu ma charakter ciągły. 

Występuje wiele algorytmów uczenia maszynowego takich jak regresja liniowa i logistyczna, drzewa decyzyjne i ich wariacje, maszyny wektorów wspierających, sztuczne sieci neuronowe oraz wiele innych. Uczenie polega na znalezieniu pewnej funkcji dopasowującej do przyjmowanego na wejściu zestawu (wektora) cech (zmiennych, kolumn) pewną odpowiedź (predykcję), która minimalizuje zadaną funkcję straty. Jej rolę w przypadku regresji często pełni błąd średniokwadratowy a w przypadku klasyfikacji np. entropia krzyżowa \angterm{cross entropy} [\#PRZYPIS]. 
Różne algorytmy szukają przy tym funkcji dopasowującej należącej do różnych klas funkcji: przykładowo klasyczne drzewa decyzyjne przeszukują tylko przestrzeń funkcji dających się opisać skończonym zbiorem reguł "jeśli - to" \angterm{if - else}.

\vspace{5em}
Tabelaryczne dane:
W pracy zamiennie używanymi terminami są wiersze i przykłady

ToDo: funkcja f(X)=y, funkcja straty, wiele algos - różne klasy funkcji, terminologia: wiersze=przyklady, kolumny = zmienne trenujace/uczace = cechy
, output = predykcja = odpowiedz algorytmu = tlo/sygnal
NN: szczegolnie dobre w nieustrukturyzowanych danych

procedura-podzial na train/val/test



