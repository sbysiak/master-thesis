\clearpage
\section{Uczenie maszynowe}
\label{sec:ml}

Uczenie maszynowe jest bardzo szerokim i obecnie dynamicznie się rozwijającym obszarem nauki. Występuje w wielu odmianach łącząc w sobie w zależności od wariantu wiele dziedzin takich jak matematyka (statystyka, algebra) informatyka (algorytmika, teoria informacji) a także elementy robotyki i sterowania. 
Dziedzinami, w których jest najczęściej wykorzystywane są min. widzenie maszynowe, przetwarzanie języka naturalnego, autonomiczne roboty i pojazdy, systemy decyzyjno - eksperckie, optymalizacyjne oraz rekomendacyjne.

W tej pracy wykorzystywana jest gałąź uczenia maszynowego nazywana uczeniem nadzorowanym lub "uczeniem z nauczycielem" \angterm{supervised learning}, gdzie uczenie występuje na podstawie poprawnie oznaczonych przykładów. Terminami bliskoznacznymi dla tak rozumianego uczenia maszynowego są uczenie statystyczne \angterm{statistical learning} i rozpoznawanie wzorców \angterm{patterm recognition}.

Problem identyfikacji dżetów jest klasycznym przykładem zagadnienia klasyfikacji, gdzie poprawna odpowiedź jest jedną ze skończonej ilości opcji (klas) w przeciwieństwie do regresji, gdzie szukana odpowiedź algorytmu ma charakter ciągły. 

Występuje wiele algorytmów uczenia maszynowego takich jak regresja liniowa i logistyczna, drzewa decyzyjne i ich wariacje, maszyny wektorów wspierających, sztuczne sieci neuronowe oraz wiele innych \cite{kotsiantis2007supervised}, \cite{wolter2012metody}. 
Uczenie polega na znalezieniu pewnej funkcji dopasowującej do przyjmowanego na wejściu zestawu (wektora) cech (zmiennych, kolumn) pewną odpowiedź (predykcję), która minimalizuje zadaną funkcję straty. Jej rolę w przypadku regresji często pełni błąd średniokwadratowy a w przypadku klasyfikacji np. entropia krzyżowa \angterm{cross entropy} \footnote{$J = -\frac{1}{N} \sum\limits_{i=1}^N \left[y_i \log\hat{y_i} + (1-y_i)\log(1-\hat{y_i}) \right]$, gdzie $y_i$ to prawidłowa klasa $i$-tego przykładu a $\hat{y_i}$ to predykcja algorytmu}. 
Różne algorytmy szukają przy tym funkcji dopasowującej należącej do różnych klas funkcji: przykładowo klasyczne drzewa decyzyjne przeszukują tylko przestrzeń funkcji dających się opisać skończonym zbiorem reguł "jeśli -- to" \angterm{if -- else}.

W pracy wykorzystane zostały dwa rodzaje algorytmów: wzmacniane drzewa decyzyjne oraz sieci neuronowe.
\FloatBarrier
\subsection{Wzmacniane drzewa decyzyjne}
\label{subsec:drzewa}

Wzmacniane drzewa decyzyjne są jednym z rozwinięć klasycznego algorytmu drzewa decyzyjnego. 
Pojedyncze drzewo decyzyjne dzieli przestrzeń cech uczących przy pomocy prostopadłych cięć, na mniejsze/większe niż zadana wartość w przypadku zmiennej ciągłej lub na należące/nie należące do danej klasy w przypadku zmiennej kategorycznej. Każdy podział, nazywany węzłem, daje dwie gałęzie, które można dalej niezależnie dzielić aż do ostatniego poziomu (liści). Kolejne podziały wybierane są tak, aby zbiory przykładów wpadające do poszczególnych gałęzi były jak najbardziej jednorodne. Stosuje się różne miary jednorodności takie tak: \textit{indeks Gini}

Drzewa decyzyjne są często łączone w komitety klasyfikatorów \angterm{ensemble methods}. Wiele "słabych" klasyfikatorów jest łączonych w jeden "silny" na dwa sposoby: \textit{bagging} oraz \textit{boosting} (wzmacnianie), które są często ze sobą porównywane.

\textit{Bagging} -- w zastosowaniu dla drzew decyzyjnych nazywany algorytmem lasów losowych \angterm{random forest} - polega na wytrenowaniu wielu drzew, każdego na podstawie $N$ przykładów losowo wylosowanych z powtórzeniami spośród $N$-licznego zbioru treningowego. Dodatkowo, do uczenia każdego drzewa używa się tylko podzbioru wszystkich cech uczących. Końcową predykcję algorytmu otrzymuje się poprzez "głosowanie" wszystkich drzew z odpowiednimi wagami.

\textit{Boosting} -- wzmacniane drzewa decyzyjne \angterm{boosted decision trees} -- jest metodą podobną do \textit{baggingu}. Główną różnicą jest zwiększanie wag przykładom uczącym, które przez poprzednie drzewo zostały źle zaklasyfikowane -- każde kolejne drzewo koncentruje się bardziej na poprawie błędów poprzednich drzew. Widać tu kolejną ważną cechę odróżniającą obie metody: \textit{boosting} jest algorytmem sekwencyjnym podczas gdy \textit{bagging} daje się trywialnie zrównoleglić (każde drzewo trenowane jest w osobnym wątku).

W niniejszej pracy wykorzystano wzmacniane drzewa decyzyjne zaimplementowane w bibliotece \texttt{XGBoost}.

\subsection{Sieci neuronowe}
\label{subsec:nn}

Sieci neuronowe są szczególnym algorytmem uczenia maszynowego. Występują w bardzo wielu odmianach i są wykorzystywane w rozwiązywaniu szerokiej gamy problemów. Nawet bardzo pobieżny opis sieci neuronowych wymaga dużo więcej miejsca niż kilka stron przewidzianych w tej pracy. Zainteresowanego czytelnika odsyłam do: [\#REF]x100. (Klasyfikację i opis typów sieci neuronowych można znaleźć w [\#REF].) Tu przedstawione zostaną wyłącznie wybrane zagadnienia mające ścisły związek z pracą, używane będą terminy, których znaczenie wyjaśniane jest w podanych źródłach.

W niniejszej pracy, wykorzystane zostały dwa rodzaje sieci neuronowych: w pełni połączone \angterm{fully connected NN -- FC NN} oraz konwolucyjne \angterm{convolutional NN -- ConvNets, CNN}. 

ToDo: Rys. (własne?) 

opis hiperparametrów (na podstawie obrazka?)

Conv - opis na 2d, zwykle obrazki, tu: 1D np. szeregi czasowe
http://www.deeplearningbook.org/contents/convnets.html - str. 329-334

różnice - lokalne połączenia, dzielenie wag, mniejsza liczba wag w Conv



\FloatBarrier
\subsection{Dyskusja użycia dwóch algorytmów}
\label{subsec:dyskusja-2algo}

Użycie więcej niż jednego algorytmu ma wiele zalet. 
Po pierwsze daje możliwość porównania wyników. Pozwala to na oszacowanie błędu \textit{bayesowskiego} (najniższego możliwego do osiągnięcia przez jakikolwiek algorytm błędu). Jest to bardzo ważne w sytuacji, gdy nie dysponuje się innym oszacowaniem tego błędu (w wielu problemach naturalnych dla człowieka jak rozpoznawanie obiektów na obrazkach jest nim błąd ludzki lub też błąd popełniany przez zespół ekspertów w bardziej zaawansowanych zastosowaniach).

Po drugie, wykorzystane zostały dwa algorytmy mocno różniące się w swojej naturze, co pozwala wykorzystać cechy każdego z nich w analizie: przykładowo sieci neuronowe dobrze radzą sobie z nieustrukturyzowanymi danymi -- potrafią tworzyć wysoko poziomowe cechy na podstawie niskopoziomowego wejścia (np. położenia oka na zdjęciu twarzy na podstawie pixeli). Są natomiast trudne w interpretacji i często traktowane są jako tzw. "czarne skrzynki" \angterm{black box}. Oprócz tego, liczba możliwych konfiguracji sieci jest ogromna i przez to niemożliwe jest stwierdzenie czy wykorzystane zostały pełne ich możliwości.

Z kolei drzewa decyzyjne posiadają stosunkowo niewielką liczbę parametrów, a ich trenowanie jest bardzo szybkie co pozwala na ich ekstensywne przeszukiwanie i otrzymanie wyników, które można uznać za optymalne dla tego algorytmu.
Ponadto, w przypadku drzew istnieją niewymagające dodatkowych obliczeń miary użyteczności poszczególnych zmiennych, co daje wgląd w działanie algorytmu i poprawia intuicyjne zrozumienie jego predykcji.






