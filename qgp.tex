\section{Fizyka dżetów cząstek}


\subsection{Chromodynamika kwantowa}

Chromodynamika kwantowa \angterm{Quantum Chromodynamics -- QCD} to kwantowa teoria pola opisująca oddziaływania silne \cite{perkins_2005}. Wprowadza ona dla kwarków nową liczbę kwantową nazywaną kolorem lub ładunkiem kolorowym, który jest odpowiednikiem ładunku elektrycznego w elektrodynamice kwantowej \angterm{Quantum Electrodynamics -- QED}, ale w przeciwieństwie do niego może przyjmować 3 różne wartości (i trzy antywartości dla antykwarków). Elementarne oddziaływania w obu teoriach przenoszone są przez bezmasowe bozony pośredniczące: w \textit{QED} jest to elektrycznie obojętny foton a w \textit{QCD} gluony, które występują w 8 odmianach i są kolorowo naładowane, przez co możliwe jest oddziaływanie zachodzące między dwoma gluonami. Kwarki i gluony zbiorczo nazywane są partonami.

Próżnia, w rozumieniu klasycznym będąca zupełnie pusta, w teoriach kwantowych wypełniona jest pojawiającymi i znikającymi wirtualnymi cząstkami. Cząstki te ekranują ładunek próbny umieszczony w kwantowej próżni, wywołując zjawisko polaryzacji próżni (analogiczne do polaryzacji dielektryków), które efektywnie zmniejsza pole wytwarzane przez ten ładunek. 
Siła tego efektu zależy od liczby ekranujących cząstek, czyli pośrednio od skali odległości. Skala ta wyznaczona jest przez długości fali próbkującej cząstki, zatem także jej energię (im większa energia, tym mniejsza długość fali i mniejsza ilość ekranujących cząstek obserwowanych w pobliżu rzeczywistego ładunku, zatem tym słabszy efekt ekranowania i większy efektywny ładunek). Prowadzi to do zależnej od energii stałej sprzężenia $\alpha$, którą nazywamy  efektywną lub biegnącą stałą sprzężenia \angterm{running coupling constant}. 


\begin{figure}[h]
	\centering
	\includegraphics[width=0.8\textwidth]{loop-feynman-diagrams.png}
	\caption{Diagramy Feynmana opisujące polaryzację próżni w \textit{QED} (lewy) i \textit{QCD} (środkowy i prawy). Rysunki lewy i środkowy są swoimi odpowiednikami w tych dwóch teoriach, natomiast prawy, w którym oddziałują jedynie bozony pośredniczące nie ma swojego odpowiednika w \textit{QED}. \imgsrc{\cite{perkins_2005}}}
	\label{fig:feynman-diagrams}
\end{figure}

Zarówno pary elektron-pozyton jak i kwark-antykwark działają ekranująco na kolejno ładunek elektryczny i kolorowy. Jednak jak zostało to już wspomniane, w przypadku \textit{QCD} możliwe jest także samooddziaływanie gluonów, przez co dopuszczalne są diagramy Feynmana jak ten przedstawiony na Rys. \ref{fig:feynman-diagrams} po prawej. Pętle gluonowe działają anty-ekranująco -- zwiększają efektywną wartość silnej stałej sprzężenia, ponadto jest to efekt dominujący nad przyczynkiem od par kwark-antykwark, co sprawia że zależność biegnącej stałej sprzężenia w \textit{QCD} jest odwrotna i dużo silniejsza niż w przypadku \textit{QED}. Wartość $\alpha_{em}$ maleje od wartości $\frac{1}{128}$ przy energiach ok. 90 GeV do $\frac{1}{137}$ przy energii bliskiej zeru, co oznacza zmianę o kilka procent. 
Tymczasem $\alpha_S$ rośnie w miarę zbliżania się do niskich energii od wartości $\alpha_S\lesssim 0.1$ dla $E \gtrsim 100$ GeV do $\alpha_S > 1$ dla energii poniżej 200 MeV (por. Rys. \ref{fig:running-coupling-const}). Prowadzi to do dwóch zjawisk charakterystycznych dla chromodynamiki kwantowej: 
\begin{itemize}
	\item asymptotyczna swoboda \angterm{asymptotic freedom} \cite{Gross:1973id}, \cite{Politzer:1973fx} -- dla wysokich energii ($\gtrsim 100$ GeV) silna stała sprzężenia jest mała (w tym zakresie energii możliwe jest stosowanie rachunku perturbacyjnego) i kwarki wewnątrz hadronów zachowują się jak cząstki quasi-swobodne.
	\item uwięzienie koloru \angterm{colour confinement} -- przy zwiększaniu odległości między partonami siła oddziaływania rośnie do nieskończoności, dlatego nigdy nie obserwuje się swobodnych cząstek obdarzonych ładunkiem kolorowym a jedynie związane w kolorowo obojętne hadrony.
\end{itemize}


\begin{figure}[h]
	\centering
	\includegraphics[trim={0cm 1.5cm 0cm 1.5cm},clip,width=0.6\textwidth]{asq-2015.eps}
	\caption{Zależność silnej stałej sprzężenia od przekazu czteropędu. \imgsrc{\cite{Patrignani:2016xqp}}.}
	\label{fig:running-coupling-const}
\end{figure}


\subsection{Plazma kwarkowo-gluonowa}
\label{subsec:qgp}

Odkrycie asymptotycznej swobody pozwoliło na sprawdzenie przewidywań \textit{QCD} w warunkach bardzo wysokich temperatur oraz gęstości. 
Dla wystarczająco dużych gęstości hadrony zaczynają na siebie zachodzić, prowadząc do stworzenia stanu, w którym poszczególne hadrony przestają być odróżnialne \cite{Collins:1974ky}. 
Zasugerowane zostało także istnienie przejścia fazowego w temperaturze porównywalnej z masą pionów oraz w temperaturze niższej, ale przy odpowiednio dużych gęstościach \cite{Cabibbo:1975ig}. Stan materii powstały po osiągnięciu, któregoś z tych warunków nazywany jest plazmą kwarkowo-gluonową \angterm{quark-gluon plasma -- QGP}.
Obecnie przewiduje się, że materia w takim stanie istniała w pierwszych ułamkach sekund po Wielkim Wybuchu \cite{Boyanovsky:2006bf} oraz, że może się znajdować w w jądrach gwiazd neutronowych \cite{Alford:2013pma}.

Obecnie aby uzyskać dostęp do materii w stanie plazmy kwarkowo-gluonowej potrzebne są wysokoenergetyczne zderzenia cząstek. Powszechnie mówi się o niej w kontekście zderzeń ciężkich jonów, chociaż istnieją także prace doszukujące się obecności \textit{QGP} w mniejszych systemach np. w zderzeniach proton-proton \cite{Khachatryan:2016txc}, \cite{ALICE:2017jyt}. 

Cechą charakterystyczną \textit{QGP} jest obecność wolnych kwarków i gluonów. Ze względu na uwięzienie koloru w każdym innym stanie materii są one zawsze związane i tworzą hadrony. 
Wolne kwarki i gluony powstające w zderzeniach muszą zatem przejść przez proces hadronizacji, w którym  rekombinują one ze spontanicznie wytwarzanymi nowymi partonami, tworząc hadrony. W wyniku tego procesu, z każdego partonu obecnego w początkowym etapie zderzenia może powstać wiele cząstek poruszających się podobnym kierunku, tworząc stożek z wierzchołkiem blisko punktu interakcji wiązek. Taki stożek skolimowanych cząstek nazywany jest dżetem cząstek.




