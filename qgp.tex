\section{Fizyka}
\subsection{Plazma kwarkowo-gluonowa}
\label{subsec:qgp}

Plazma kwarkowo--gluonowa \angterm{quark-gluon plasma -- QGP} to stan materii, który istniał w pierwszych ułamkach sekund po Wielkim Wybuchu. Przewiduje się, że materia w takim stanie obecna jest także w jądrach gwiazd neutronowych \cite{Andronic:2014zha}.

Obecnie aby uzyskać dostęp do materii w stanie plazmy kwarkowo-gluonowej potrzebne są wysokoenergetyczne zderzenia cząstek. Powszechnie mówi się o niej w kontekście zderzeń ciężkich jonów, chociaż istnieją także prace doszukujące się obecności \textit{QGP} w mniejszych systemach np. w zderzeniach proton-proton \cite{Khachatryan:2016txc}, \cite{ALICE:2017jyt}. 

\textit{QGP} jest stanem o ekstremalnej gęstości i temperaturze. Jego cechą charakterystyczną jest obecność wolnych kwarków i gluonów (zbiorczo nazywanych partonami). 
W każdym innym stanie materii są one zawsze związane i tworzą hadrony. Zjawisko to, zwane uwięzieniem koloru \angterm{color confinement} uniemożliwia obserwację cząstek obdarzonych ładunkiem kolorowym (a takimi są kwarki i gluony) w stanie wolnym.
Wolne kwarki i gluony powstające w zderzeniach muszą zatem przejść przez proces hadronizacji, w którym  rekombinują one ze spontanicznie wytwarzanymi nowymi partonami, tworząc hadrony. W wyniku tego procesu, z każdego partonu obecnego w początkowym etapie zderzenia może powstać wiele cząstek poruszających się podobnym kierunku, tworząc stożek z wierzchołkiem blisko punktu interakcji wiązek. Taki stożek skolimowanych cząstek nazywany jest dżetem cząstek.




