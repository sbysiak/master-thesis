\section{Fizyka w}


\subsection*{Chromodynamika kwantowa}

Chromodynamika kwantowa \angterm{Quantum Chromodynamics -- QCD} to kwantowa teoria pola opisująca oddziaływania silne. Wprowadza ona dla kwarków nową liczbę kwantową nazywaną kolorem lub ładunkiem kolorowym, który jest odpowiednikiem ładunku elektrycznego w elektrodynamice kwantowej \angterm{Quantum Electrodynamics -- QED}, ale w przeciwieństwie do niego może przyjmować 3 różne wartości (i trzy antykolory dla antykwarków). Elementarne oddziaływania w obu teoriach przenoszone są przez bezmasowe bozony pośredniczące: w \textit{QED} jest to elektrycznie obojętny foton a w \textit{QCD} gluony, które występują w 8 odmianach i są kolorowo naładowane, przez co możliwe jest oddziaływanie zachodzące między dwoma gluonami.

Próżnia, w rozumieniu klasycznym będąca zupełnie pusta, w teoriach kwantowych wypełniona jest pojawiającymi i znikającymi wirtualnymi cząstkami. Cząstki te ekranują ładunek próbny umieszczony w kwantowej próżni, wywołując zjawisko polaryzacji próżni (analogiczne do polaryzacji dielektryków), które efektywnie zmniejsza pole wytwarzane przez ten ładunek. 
Siła tego zjawiska zależy od liczby ekranujących cząstek, czyli pośrednio od skali odległości. Skala ta wyznaczona jest przez długości fali próbkującej cząstki, zatem także jej energii (im większa energia, tym mniejsza długość fali i mniejsza ilość ekranujących cząstek obserwowanych w pobliżu rzeczywistego ładunku, zatem tym słabszy efekt ekranowania i większy efektywny ładunek). Prowadzi to do zależnej od energii stałej sprzężenia $\alpha$, którą nazywamy  efektywną lub biegnącą stałą sprzężenia \angterm{running coupling constant}. 


\begin{figure}[h]
	\centering
	\includegraphics[width=0.8\textwidth]{loop-feynman-diagrams.png}
	\caption{Diagramy Feynmana opisujące polaryzację próżni w \textit{QED} (lewy) i \textit{QCD} (środkowy i prawy). Rysunki lewy i środkowy są swoimi odpowiednikami w tych dwóch teoriach, natomiast prawy, w którym oddziałują jedynie bozony pośredniczące nie ma swojego odpowiednika w \textit{QED}. Źródło: \cite{perkins_2005}.}
	\label{fig:feynman-diagrams}
\end{figure}

Zarówno pary elektron-pozyton jak i kwark-antykwark działają ekranująco na kolejno ładunek elektryczny i kolorowy. Jednak jak zostało to już wspomniane, w przypadku \textit{QCD} możliwe jest także samooddziaływanie gluonów, przez co dopuszczalne są diagramy Feynmana jak ten przedstawiony na Rys. \ref{fig:feynman-diagrams} po prawej. Pętle gluonowe działają anty-ekranująco -- zwiększają efektywną wartość stałej sprzężenia, ponadto jest to efekt dominujący nad przyczynkiem od par kwark-antykwark, co sprawia że zależność biegnącej stałej sprzężenia w \textit{QCD} jest odwrotna i dużo silniejsza niż w przypadku \textit{QED}. Wartość $\alpha_{em}$ rośnie od wartości $\frac{1}{137}$ przy  energii bliskiej zeru do $\frac{1}{128}$ przy energiach ok. 90 GeV, co oznacza zmianę o kilka procent. Tymczasem $\alpha_S$ rośnie asymptotycznie w miarę zbliżania się do niskich energii (por. Rys. \ref{fig:running-coupling-const}) osiągając wartości $\alpha_S > 1$ dla energii poniżej 200 MeV. Prowadzi to do dwóch zjawisk charakterystycznych dla chromodynamiki kwantowej: 
\begin{itemize}
	\item asymptotyczna swoboda -- dla wysokich energii ($\gtrsim 100$ GeV) silna stała sprzężenia jest mała ($\lesssim 0.1$) i kwarki wewnątrz hadronów zachowują się jak cząstki quasi-swobodne. W tym zakresie energii możliwe jest stosowanie rachunku perturbacyjnego.
	\item uwięzienie koloru -- przy zwiększaniu odległości między partonami siła oddziaływania rośnie do nieskończoności, dlatego nigdy nie obserwuje się swobodnych cząstek obdarzonych ładunkiem kolorowym.
\end{itemize}


\begin{figure}[h]
	\centering
	\includegraphics[width=0.6\textwidth]{running-coupling-const.jpg}
	\caption{Zależność silnej stałej sprzężenia $\alpha_S$ od przekazu czteropędu $Q$.}
	\label{fig:running-coupling-const}
\end{figure}


\subsection{Plazma kwarkowo-gluonowa}
\label{subsec:qgp}

Plazma kwarkowo--gluonowa \angterm{quark-gluon plasma -- QGP} to stan materii, który istniał w pierwszych ułamkach sekund po Wielkim Wybuchu. Przewiduje się, że materia w takim stanie obecna jest także w jądrach gwiazd neutronowych \cite{Andronic:2014zha}.

Obecnie aby uzyskać dostęp do materii w stanie plazmy kwarkowo-gluonowej potrzebne są wysokoenergetyczne zderzenia cząstek. Powszechnie mówi się o niej w kontekście zderzeń ciężkich jonów, chociaż istnieją także prace doszukujące się obecności \textit{QGP} w mniejszych systemach np. w zderzeniach proton-proton \cite{Khachatryan:2016txc}, \cite{ALICE:2017jyt}. 

\textit{QGP} jest stanem o ekstremalnej gęstości i temperaturze. Jego cechą charakterystyczną jest obecność wolnych kwarków i gluonów (zbiorczo nazywanych partonami). 
W każdym innym stanie materii są one zawsze związane i tworzą hadrony. Zjawisko to, zwane uwięzieniem koloru \angterm{color confinement} uniemożliwia obserwację cząstek obdarzonych ładunkiem kolorowym (a takimi są kwarki i gluony) w stanie wolnym.
Wolne kwarki i gluony powstające w zderzeniach muszą zatem przejść przez proces hadronizacji, w którym  rekombinują one ze spontanicznie wytwarzanymi nowymi partonami, tworząc hadrony. W wyniku tego procesu, z każdego partonu obecnego w początkowym etapie zderzenia może powstać wiele cząstek poruszających się podobnym kierunku, tworząc stożek z wierzchołkiem blisko punktu interakcji wiązek. Taki stożek skolimowanych cząstek nazywany jest dżetem cząstek.




