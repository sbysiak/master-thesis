\subsection{Dżety}
\label{subsec:dzet-def}

Przedstawiona powyżej definicja dżetu nie jest precyzyjna z punktu widzenia pracy eksperymentalnej. W detektorze obserwuje się tylko cząstki w stanie końcowym, nie jest znana natomiast ich historia (tj. parton, z którego powstały), nie jest zatem możliwe przyporządkowanie cząstki według jej pochodzenia. W związku z tym, konieczne jest użycie algorytmu klasteryzującego, dostającego na wejściu tylko obserwowalne eksperymentalnie cząstki. 
To jakie dżety zostaną zaobserwowane w danym zdarzeniu zależy od użytego algorytmu. Oznacza to, że precyzyjną definicję dżetu stanowi algorytm klasteryzujący wraz z zestawem parametrów. Obecnie najpowszechniej stosowanym algorytmem jest algorytm \textit{anti-kt} \cite{Cacciari:2008gp}.



%Jak zostało to już wcześniej zasygnalizowane, w eksperymencie nie ma się bezpośredniego dostępu do informacji na temat partonów, dlatego również klasyfikowanie (nazywane także identyfikacją) dżetów może odbywać się wyłącznie na podstawie cząstek mierzonych w detektorze -- zwykle obecności hadronu zawierającego odpowiedni kwark w zadanej odległości od osi dżetu (przykładowo obecność hadronu \textit{B} oznacza, że dżet będzie zaklasyfikowany jako dżet \textit{b}).


Eksperymentalne ograniczenia związane z obserwacją tylko końcowego stanu oddziaływań nie występują w analizie danych z symulacji Monte Carlo (MC), gdzie  ma się dostęp do pełnej informacji na temat historii każdej cząstki. 
Można pomyśleć, że daje to możliwość lepszej klasteryzacji dżetów. 
Rekonstrukcja przy użyciu informacji o partonach - matkach sprawia jednak, że definicja dżetu wykorzystana w badaniach danych symulacyjnych jest inna od tej wykorzystywanej w eksperymencie. Tracona jest przez to cecha odpowiedniości między obiektami nazywanymi dżetami w symulacji  i w eksperymencie, która to cecha jest niewątpliwie jedną z podstawowych wymagań stawianych przed dobrą symulacją.

Dżety stanowią ważny element w badaniach plazmy kwarkowo-gluonowej.
Dają one pośredni wgląd we właściwości \textit{QGP} na podstawie jej wpływu na oddziałujące z nią partony. Przykładową obserwablą mierzoną w zderzeniach ciężkich jonów jest czynnik modyfikacji jądrowej \angterm{nuclear modification factor}, który jest miarą strat energii przez parton przechodzący przez medium.


Oprócz globalnego wpływu medium na dżety, analizuje się także różnice między dżetami pochodzącymi z gluonów oraz kwarków o różnych zapachach \angterm{flavours}. Modele teoretyczne przewidują między innymi większe straty energii w wyniku interakcji z \textit{QGP} dla dżetów gluonowych niż kwarkowych \cite{Salgado:2003gb} oraz zależność strat energii od masy partonu \cite{Dokshitzer:2001zm} -- w tym przypadku precyzyjne pomiary rozróżniające typy dżetów pozwalają lepiej zrozumieć mechanizm odpowiadający za straty energii przez partony. Zagadnienie rozpoznania z jakiego rodzaju partonu powstał dany dżet, nazywane jest identyfikacją lub tagowaniem dżetu.


\subsection{Identyfikacja dżetów \textit{b}}
\label{subsec:b-dzety}

Poza badaniami właściwości \textit{QGP}, szczególne znaczenie ma identyfikacja dżetów pochodzących z ciężkich kwarków: \textit{b} i \textit{c}.
Są one ważnym elementem w poszukiwaniu łamania symetrii \textit{CP} w rozpadach hadronów B i D oraz innych sygnatur tzw. \textit{Nowej Fizyki} wykraczającej poza ramy Modelu Standardowego. 
Kwarki \textit{piękne} pojawiają się także często w kanałach rozpadu cząstek takich jak bozon Higgsa i kwark \textit{t}.

Identyfikacja dżetów \textit{b} jest sporym wyzwaniem ze względu na zdecydowanie częściej występujące dżety lekkie, tj. powstałe z hadronizacji kwarków \textit{u,d,s} oraz gluonów.
Rozpoznawanie dżetów \textit{b} bazuje na charakterystycznych właściwościach hadronów zawierających kwark piękny: relatywnie długim czasie życia oraz (w mniejszym stopniu) na ich pół-leptonowym rozpadach o względnej częstości rozpadu w tym kanale \angterm{branching ratio} na poziomie 10\%.

\textbf{przegląd algorytmów używanych w identyfikacji b-jetów: TBD} 

\subsection{Eksperyment ALICE}

TBD

%PYTHIA6 manual:
%"Parton distributions are most familiar for hadrons, such as the proton, which are
%inherently composite objects, made up of quarks and gluons. Since we do not understand
%QCD, a derivation from first principles of hadron parton distributions does not yet exist,
%although some progress is being made in lattice QCD studies. It is therefore necessary
%to rely on parameterizations, where experimental data are used in conjunction with the
%evolution equations for the Q2 dependence, to pin down the parton distributions. Several
%different groups have therefore produced their own fits, based on slightly different sets of
%data, and with some variation in the theoretical assumptions"

