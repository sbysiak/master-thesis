\subsection{Dżety}
\label{subsec:dzet-def}

Przedstawiona powyżej definicja dżetu nie jest precyzyjna z punktu widzenia pracy eksperymentalnej. W detektorze obserwuje się tylko cząstki w stanie końcowym, nie jest zatem możliwe przyporządkowanie cząstki do dżetu według jej pochodzenia. W związku z tym, konieczne jest użycie algorytmu klasteryzującego, dostającego na wejściu tylko obserwowalne eksperymentalnie cząstki. 
To jakie dżety zostaną zaobserwowane w danym zdarzeniu zależy od użytego algorytmu. Oznacza to, że precyzyjną definicję dżetu stanowi algorytm klasteryzujący wraz z zestawem parametrów. Obecnie najpowszechniej stosowanym algorytmem jest algorytm \textit{anti-kt} \cite{Cacciari:2008gp}.



%Jak zostało to już wcześniej zasygnalizowane, w eksperymencie nie ma się bezpośredniego dostępu do informacji na temat partonów, dlatego również klasyfikowanie (nazywane także identyfikacją) dżetów może odbywać się wyłącznie na podstawie cząstek mierzonych w detektorze -- zwykle obecności hadronu zawierającego odpowiedni kwark w zadanej odległości od osi dżetu (przykładowo obecność hadronu \textit{B} oznacza, że dżet będzie zaklasyfikowany jako dżet \textit{b}).


Eksperymentalne ograniczenia związane z obserwacją tylko końcowego stanu oddziaływań nie występują w analizie danych z symulacji Monte Carlo (MC), gdzie ma się dostęp do pełnej informacji na temat historii każdej cząstki. 
Nie należy jednak używać jej do klasteryzacji dżetów, gdyż utracona zostałaby cecha odpowiedniości między obiektami nazywanymi dżetami w symulacji i w eksperymencie, która to cecha jest niewątpliwie jedną z podstawowych wymagań stawianych przed dobrą symulacją. Właściwym podejściem jest rekonstrukcja dżetów przy pomocy takiego samego algorytmu jak w przypadku danych eksperymentalnych.

Dżety wykorzystuje się w badaniach plazmy kwarkowo-gluonowej.
Dają one pośredni wgląd we właściwości \textit{QGP} na podstawie wpływu jaki wywiera na oddziałujące z nią partony. 
Przykładową obserwablą mierzoną w zderzeniach ciężkich jonów jest czynnik modyfikacji jądrowej $R_{AA}$ \angterm{nuclear modification factor}, który jest miarą strat energii przez parton przechodzący przez medium. Jest to stosunek pędowych rozkładów dżetów cząstek zmierzonych w zderzeniach ciężkich jąder oraz w zderzeniach pp (przemnożonych przez liczbę binarnych zderzeń nukleon-nukleon przewidywanych przez model teoretyczny). Odchylenia od wartości 1 dla wysokich $p_T$ są oznaką modyfikacji pędów dżetów przez gęste medium (w stosunku do zderzeń pp gdzie \textit{QGP} nie powstaje), jest to tzw. tłumienie dżetów \angterm{jet quenching}. Rezultaty pomiarów $R_{AA}$ dostępne są w \cite{Khachatryan:2016odn} (CSM) i \cite{Abelev:2012hxa} (ALICE).

Oprócz globalnego wpływu medium na dżety, analizuje się także różnice między dżetami pochodzącymi z gluonów oraz kwarków o różnych zapachach \angterm{flavours}. Modele teoretyczne przewidują między innymi większe straty energii w wyniku interakcji z \textit{QGP} dla dżetów gluonowych niż kwarkowych \cite{Salgado:2003gb} oraz zależność strat energii od masy partonu \cite{Dokshitzer:2001zm} -- w tym przypadku precyzyjne pomiary rozróżniające typy dżetów pozwalają lepiej zrozumieć mechanizm odpowiadający za straty energii przez partony. Zagadnienie rozpoznania z jakiego rodzaju partonu powstał dany dżet, nazywane jest identyfikacją lub tagowaniem dżetu.
Ważną rolę w studiowaniu tego problemu odgrywają symulacje MC, które pozwalają określić wydajności poszczególnych technik tagowania dżetów na podstawie znajomości kanału produkcji każdej symulowanej cząstki.

\subsection{Dżety \textit{b}}
\label{subsec:b-dzety}

\subsubsection{Właściwości}
Poza badaniami właściwości \textit{QGP}, szczególne znaczenie ma identyfikacja dżetów pochodzących z ciężkich kwarków: \textit{b} i \textit{c}.
Są one ważnym elementem w poszukiwaniu łamania symetrii \textit{CP} w rozpadach hadronów B i D oraz innych sygnatur tzw. \textit{Nowej Fizyki} wykraczającej poza ramy Modelu Standardowego. 
Kwarki \textit{piękne} pojawiają się także często w kanałach rozpadu cząstek takich jak bozon Higgsa i kwark \textit{t}.

Identyfikacja dżetów \textit{b} jest sporym wyzwaniem ze względu na zdecydowanie częściej występujące dżety lekkie, tj. powstałe z hadronizacji kwarków \textit{u,d,s} lub gluonów.
Rozpoznawanie dżetów \textit{b} bazuje na charakterystycznych właściwościach hadronów zawierających kwark piękny: relatywnie długim czasie życia oraz (w mniejszym stopniu) na ich półleptonowych rozpadach o względnej częstości rozpadu w tym kanale \angterm{branching ratio} na poziomie 10\%.

\subsubsection{Przegląd algorytmów używanych do identyfikacji dżetów \textit{b} na LHC} 
\label{subsubsec:przeglad-algo}
Używane w eksperymentach na LHC: ATLAS, CMS i ALICE algorytmy można podzielić na trzy kategorie: wykorzystujące wtórne wierzchołki,  informację o odległości najbliższego zbliżenia (parametrze zderzenia) cząstek tworzących dżet \angterm{Distance of Closest Approach -- DCA, Impact Parameter -- IP} oraz identyfikujące produkty półleptonowych rozpadów pięknych lub powabnych hadronów. Dokładne opisy omawianych algorytmów można znaleźć w: \cite{Aad:2015ydr}, \cite{Aaboud:2018xwy} (ATLAS), \cite{Chatrchyan:2012jua}, \cite{Sirunyan:2017ezt} (CMS), \cite{Feldkamp:2013cya}, \cite{Haake:2017dpr} (ALICE).

Najprostszym algorytmem jest dyskryminacja na podstawie istotności statystycznej (wynik pomiaru podzielony przez jego niepewność) odległości wtórnego wierzchołka od wierzchołka pierwotnego $L$. Jest to metoda wykorzystywana w  każdym z trzech wymienionych eksperymentów (por. ATLAS: algorytm SV0, CMS: algorytm SSV, ALICE). 
Może być ona rozszerzona poprzez użycie dodatkowych zmiennych opisujących wtórny wierzchołek jak na przykład jego masa, ułamek niesionej przez niego całkowitej energii dżetu (por. ATLAS: SV1) lub użycie "pseudowierzchołków" (kombinacji dwóch cząstek o dużych \textit{DCA}) w celu poprawienia wydajności detekcji o przypadki, w których wtórny wierzchołek nie został zrekonstruowany (por. CMS: CSV).

Algorytmy wykorzystujące informację o poszczególnych cząstkach mogą zasadniczo bazować albo na sumie logarytmów prawdopodobieństw pochodzenia każdej cząstki z pierwotnego wierzchołka (por. ATLAS: IP3D, CMS: JP) lub tym samym prawdopodobieństwie ale dla wybranej, np. drugiej lub trzeciej cząstki na liście posortowanej według malejącego \textit{IP} (por. ALICE i CMS: TC). Bardziej złożonym podejściem, w którym cząstki nie są traktowane jako niezależne, jest użycie rekurencyjnych sieci neuronowych (por. ATLAS: RNNIP).

Do wykorzystania półleptonowego kanału rozpadu ciężkich hadronów do identyfikacji dżetów \textit{b} w eksperymentach ATLAS (SMT) i CMS (SE, SM) użyto wzmacnianych drzew decyzyjnych trenowanych na kilku ręcznie zdefiniowanych w tym celu zmiennych takich jak pęd leptonu transwersalny względem osi dżetu.

Znaczącą poprawę zdolności predykcyjnej można uzyskać łącząc kilka różnych modeli. 
Algorytm łączący może pobierać na wejściu albo tylko predykcje klasyfikatorów niższego poziomu (CMS: cMVAv2) lub dodatkowo także ich zmienne wejściowe (ATLAS: MV2, DL1). Do scalania używane są zwykle wzmacniane drzewa decyzyjne lub sieci neuronowe.

Przykład innego podejścia zaprezentowała współpraca przy eksperymencie LHCb, gdzie wykorzystano dwa zestawy wzmacnianych drzew decyzyjnych operujących na zmiennych związanych z wtórnymi wierzchołkami. Pierwszy zapewnia separację dżetów lekkich od ciężkich a drugi odróżnia dżety \textit{b} od \textit{c}. 
Do wyboru punktu pracy, zamiast jednowymiarowego rozkładu predykcji używany jest dwuwymiarowy rozkład wag przypisany przez oba klasyfikatory \cite{Aaij:2015yqa}.


\subsection{Eksperyment ALICE}

Eksperyment ALICE \cite{Aamodt:2008zz}, \cite{Abelev:2014ffa} jest jednym z czterech największych eksperymentów na LHC. Jest on dedykowany zderzeniom ciężkich jonów (w LHC są to jony ołowiu PbPb), ale mierzone są także mniejsze systemy, tj. proton-proton pp (głównie jako referencję dla pomiarów PbPb) oraz proton-ołów p+Pb, które dostarczają także okazji do badania asymetrycznych zderzeń. Cechą charakterystyczną pomiarów ciężkojonowych jest ich znacznie większa niż w przypadku zderzeń pp krotność, tj. liczba cząstek wyprodukowana w pojedynczym zderzeniu. W przypadku zderzeń PbPb może powstawać nawet do 8000 naładowanych cząstek na jednostkę pseudopospieszności $\eta$ \angterm{pseudorapidity} \footnote{$\eta = -\ln[\tg(\frac{\theta}{2})]$, gdzie $\theta$ jest kątem między wektorem pędu cząstki a osią wiązki}. Detektor ALICE został zoptymalizowany do mierzeniach takich przypadków, jak również pod kątem rekonstrukcji i identyfikacji cząstek o szerokim zakresie pędów (100 MeV -- 100 GeV).


\begin{figure}[h]
	\centering
	\includegraphics[width=0.99\textwidth]{alice-detector.jpg}
	\caption{Schemat detectora ALICE. \imgsrc{\cite{alice-det}} }
	\label{fig:alice-detector}
\end{figure}


Detektor ALICE jest urządzeniem złożonym z wielu subdetektorów, schematycznie przedstawionych na Rys. \ref{fig:alice-detector}. Można je podzielić według pełnionej w pomiarach roli. 
ITS, TPC, TRD oraz TOF pokrywają pełen kąt azymutalny oraz zakres pseudopospieszności $|\eta| < 0.9$.
\begin{itemize}
	\item Detektory śladowe -- mierzące trajektorie cząstek zakrzywiane w polu magnetycznym o wartości $B = 0.5$ T.
	\begin{itemize}
		\item Inner Tracking System (ITS) -- zespół krzemowych detektorów śladowych znajdujący się najbliżej miejsca interakcji wiązek. Składa się on z 6 cylindrycznych warstw o promieniach od 4 do 43 cm, wykonanych w trzech różnych technologiach. Jego główną rolą jest rekonstrukcja pierwotnego oraz wtórnych wierzchołków. Bierze także udział w rekonstrukcji trajektorii i strat energetycznych cząstek, szczególnie tych niskopędowych, które nie docierają do dalej położonych detektorów.
		\item Time Projection Chamber (TPC) -- długa na 5m i o takiej średnicy komora projekcji czasowej. Jest to główny detektor śladowy ALICE, wraz z ITS służy do wyznaczania trajektorii cząstek i na ich podstawie również wierzchołków zderzenia. Elektrony uwolnione ze zjonizowanego przez poruszające się w nim naładowane cząstki gazu dryfują wzdłuż kierunku wiązki w stronę końcowych elektrod. Następnie są tam zbierane dostarczając informacji o dwóch współrzędnych toru cząstki: odległości od wiązki i kącie azymutalnym. Trzecia składowa trajektorii jest otrzymywana na podstawie czasu dotarcia elektronów do elektrod. TPC jest najwolniejszym detektorem ALICE (ze względu na ograniczający czas dryfu elektronów wynoszący $\sim$90 $\mu$s), użycie detektora tego typu podyktowane jest jego zdolnością do rozwikłania śladów tysięcy cząstek spodziewanych w centralnych zderzeniach PbPb.
		
		Znajomość toru ruchu cząstki pozwala na wyznaczenie jej pędu. Oprócz dokładnej trajektorii każdej cząstki próbkowanej do 159 razy, TPC mierzy straty energii cząstek $dE/dx$. Pozwala to na ich identyfikacje na podstawie wzoru Bethego-Blocha, najwyższą zdolność rozdzielczą TPC osiąga dla cząstek o $p_T < 1$ GeV.
	\end{itemize}
	\item detektory służące identyfikacji cząstek \angterm{particle identification -- PID}
	\begin{itemize}
		\item Transition Radiation Detector (TRD) -- detektor wykrywający promieniowanie przejścia, służy głównie do odróżniania wysokopędowych ($p_T > 1$ GeV) elektronów od pionów. Promieniowanie przejścia emitowane jest podczas przechodzenia relatywistycznych cząstek przez granicę ośrodków, jego intensywność jest proporcjonalna do czynnika Lorentza $\gamma$, co pozwala na odróżnienie cząstek o tym samym pędzie na podstawie różnicy mas (elektrony są ponad 250 razy lżejsze od pionów). TRD oprócz identyfikacji elektronów uczestniczy także w rekonstrukcji śladów wysokopędowych cząstek i może być użyty w systemach wyzwalania \angterm{trigger}.
		\item Time-Of-Flight (TOF) -- detektor czasu przelotu o zdolności rozdzielczej $\sim80$ ps. Pozwala na separację pionów i kaonów o pędach do ok. 2.5 GeV i protonów do 4 GeV.  
		\item High-Momentum Particle Identification Detector (HMPID) -- detektor typu RICH \angterm{ring-imaging Cherenkov}, wykrywający fotony emitowane podczas przejścia przez ośrodek naładowanej cząstki o prędkości większej od prędkości fazowej światła w tym ośrodku (promieniowanie Cherenkowa). Na podstawie kąta pod jakim emitowane są fotony określana jest prędkość cząstki. HMPID pozwala na identyfikację pionów, kaonów i protonów o $p_T > 1$ GeV. Pokrywa przestrzeń kątów: $\ang{1.2} < \phi < \ang{58.8}$ oraz $|\eta| < 0.6$ (5\% akceptancji TPC).
	\end{itemize}
	\item kalorymetry
	\begin{itemize}
		\item Photon Spectrometer (PHOS) -- elektromagnetyczny kalorymetr o wysokiej rozdzielczości energetycznej i przestrzennej (podzielony na kryształy o rozmiarze poprzecznym 2.2 x 2.2 cm, co odpowiada rozmiarowi w dziedzinie $\eta,~\phi$ 0.004 x 0.004). Pokrywa zakres pseudopospieszności $|\eta| < 0.12$ i kąta azymutalnego równy \ang{100}. PHOS ma za zadanie identyfikację i pomiar czteropędów fotonów, w szczególności tych niepochodzących z rozpadu innych cząstek \angterm{direct photons} oraz lekkich mezonów neutralnych (np. $\pi^0$) przez dwufotonowy kanał rozpadu.
		\item Electromagnetic Calorimeter (EMCal) -- drugi elektromagnetyczny kalorytmetr ALICE o mniejszej ziarnistości ($\Delta\eta,\Delta~\phi$ = 0.014 x 0.014 ), ale dużo większej akceptancji ($|eta| < 0.7$, $\Delta\phi = \ang{107}$).
EMCal poprawia możliwości ALICE w zakresie pomiarów tłumienia dżetów, pozwalając na wyznaczanie neutralnej składowej energii dżetów (energii niesionej przez neutralne cząstki). Dzięki innej charakterystyce dla elektronów i hadronów (elektrony typowo deponują niemal całą energię a hadrony tylko niewielką część) pozwala je odróżnić na podstawie stosunku zmierzonej w nim energii do wyznaczonego wcześniej pędu $E/p$.
EMCal może być użyty także w szybkim systemie wyzwalania, do selekcji przypadków z dżetami oraz wysokoenergetycznymi fotonami i elektronami.
	\end{itemize}
	\item Muon spectrometer -- spektrometr mionowy, złożony z dwóch pasywnych absorberów, znajdujących się między nimi 10 warstw detektora śladowego oraz komór systemu wyzwalającego na końcu. 
	Przedni absorber, gruby na 4 metry (${\sim}60 X_0$) wykonany z betonu i grafitu, zatrzymuje hadrony oraz miony o niższych energiach (np. z rozpadów pionów i kaonów). Jest on zoptymalizowany aby minimalizować rozpraszanie mionów i zapewnić ochronę pozostałych detektorów ALICE przed wtórnymi cząstkami powstałymi w jego materiale. 
Komory pozycjoczułe mają zdolność rozdzielczą ok. 100 $\mu$m, co pozwala osiągnąć wysoką rozdzielczość przy wyznaczaniu masy niezmienniczej rzędu 100 MeV/$c^2$.	Spektrometr mionowy służy głównie do mierzenia mezonów wektorowych ($\omega,\;\phi,\;J/\Psi,\;\Upsilon$) rozpadających się w kanale $\mu^+\mu^-$.
	
	\item Detektory przednie, wyznaczające min. centralność zderzeń oraz płaszczyznę reakcji.
	\begin{itemize}
		\item ZDC -- zespół czterech kalorymetrów (po dwa do pomiaru protonów i neutronów, gdyż ich tory są rozdzielane przez pole magnetyczne) mierzących energię nukleonów nieuczestniczących w zderzeniu tzw. obserwatorów, co pozwala na określenie liczby nukleonów oddziałujących, tzw. uczestników. Znajdują się one 116 m od miejsca interakcji.
		\item PMD -- detektor mierzący krotności oraz rozkład przestrzenny fotonów
		\item FMD -- krzemowy detektor paskowy mierzący precyzyjnie liczbę naładowanych cząstek w zakresie pseudopospieszności wykraczającym poza akceptancję detektora ITS.
		\item V0 -- liczniki scyntylacyjne położone po obu stronach detektora, używane w systemie wyzwalania o minimalnym obciążeniu \angterm{minimum bias trigger} -- wymóg obecności sygnału w obu detektorach pozwala na odrzucenie przypadków tła z oddziaływania wiązki protonów z resztkami gazu obecnymi w rurach próżniowych.
		\item T0 -- dostarcza dokładny czas interakcji potrzebny dla detektora TOF, pozwala także na śledzenie świetlności w czasie rzeczywistym.
	\end{itemize}
\end{itemize}





%PYTHIA6 manual:
%"Parton distributions are most familiar for hadrons, such as the proton, which are
%inherently composite objects, made up of quarks and gluons. Since we do not understand
%QCD, a derivation from first principles of hadron parton distributions does not yet exist,
%although some progress is being made in lattice QCD studies. It is therefore necessary
%to rely on parameterizations, where experimental data are used in conjunction with the
%evolution equations for the Q2 dependence, to pin down the parton distributions. Several
%different groups have therefore produced their own fits, based on slightly different sets of
%data, and with some variation in the theoretical assumptions"

