\thispagestyle{empty}
%% ------------------------ NAGLOWEK STRONY ---------------------------------
\includegraphics[height=37.5mm]{agh_nzw_a_pl_1w_wbr_cmyk.eps}\\
\rule{30mm}{0pt}
{\large \textsf{Wydział Fizyki i Informatyki Stosowanej}}\\
\rule{\textwidth}{3pt}\\
\rule[2ex]
{\textwidth}{1pt}\\
\vspace{7ex}
\begin{center}
{\LARGE \bf \textsf{Praca magisterska}}\\
\vspace{13ex}
% --------------------------- IMIE I NAZWISKO -------------------------------
{\bf \Large \textsf{Sebastian Bysiak}}\\
\vspace{3ex}
{\sf\small kierunek studiów:} {\bf\small \textsf{fizyka techniczna}}\\
\vspace{1.5ex}
%{\sf\small specjalność:} {\bf\small \textsf{fizyka komputerowa}}\\
\vspace{10ex}
%% ------------------------ TYTUL PRACY --------------------------------------
{\bf \huge \textsf{Rozwój algorytmów do identyfikacji dżetów cząstek w~pomiarach zderzeń proton-proton i~ołów-ołów przy pomocy ALICE na LHC}}\\
\vspace{14ex}
%% ------------------------ OPIEKUN PRACY ------------------------------------
{\Large Opiekun: \bf \textsf{dr Jacek Otwinowski}}\\
\vspace{22ex}
{\large \bf \textsf{Kraków, sierpień 2018}}
\end{center}
%% =====  STRONA TYTUŁOWA PRACY MAGISTERSKIEJKIEJ ====

\newpage

%% =====  TYŁ STRONY TYTUŁOWEJ PRACY MAGISTERSKIEJKIEJ ====
{\sf Oświadczam, świadomy(-a) odpowiedzialności karnej za poświadczenie nieprawdy, że
niniejszą pracę dyplomową wykonałem(-am) osobiście i samodzielnie i nie korzystałem(-am) ze źródeł innych niż wymienione w pracy.}

\vspace{14ex}

\begin{center}
\begin{tabular}{lr}
~~~~~~~~~~~~~~~~~~~~~~~~~~~~~~~~~~~~~~~~~~~~~~~~~~~~~~~~~~~~~~~~~ &
................................................................. \\
~ & {\sf (czytelny podpis)}\\
\end{tabular}
\end{center}

%% =====  TYL STRONY TYTULOWEJ PRACY MAGISTERSKIEJKIEJ ====

\newpage
\rightline{Kraków, ... sierpnia 2018}
\begin{center}
{\bf Tematyka pracy magisterskiej i praktyki dyplomowej
Sebastiana Bysiaka,
studenta V roku studiów kierunku fizyka techniczna}\\
\end{center}

Temat pracy magisterskiej:
{\bf Rozwój algorytmów do identyfikacji dżetów cząstek w pomiarach zderzeń proton-proton i ołów-ołów przy pomocy ALICE na LHC}\\

\begin{tabular}{rl}

Opiekun pracy:                  & dr Jacek Otwinowski\\
Recenzenci pracy:               & ......................\\
Miejsce praktyki dyplomowej:    & IFJ PAN, Kraków\\
\end{tabular}

\begin{center}
{\bf Program pracy magisterskiej i praktyki dyplomowej}
\end{center}

\begin{enumerate}
\item Omówienie realizacji pracy magisterskiej z opiekunem.
\item Zebranie i opracowanie literatury dotyczącej tematu pracy.
\item Praktyka dyplomowa:
\begin{itemize}
\item zapoznanie się z idea
\item uczestnictwo w warsztatach z używanych w ALICE środowisk programistycznych,
\item sporządzenie sprawozdania z praktyki.
\end{itemize}
\item Zebranie i przetworzenie danych do formatu używanego w analizie.
\item Testowanie różnych architektur sieci neuronowych.
\item Trenowanie modeli.
\item Omówienie uzyskanych wyników.
\item Analiza właściwości modeli.
\item Opracowanie redakcyjne pracy.
\end{enumerate}


\noindent
Termin oddania w dziekanacie: ... września 2018\\[1cm]

\begin{center}
\begin{tabular}{lcr}
.............................................................. & ~~~ &
.............................................................. \\
(podpis kierownika katedry) & & (podpis opiekuna) \\
\end{tabular}
\end{center}


%\linespread{1.3}
\selectfont

\clearpage
Recenzja 1
\clearpage
Recenzja 2
\clearpage