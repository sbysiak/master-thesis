\subsection{Sieci neuronowe}
\label{subsec:nn}

Sieci neuronowe są szczególnym algorytmem uczenia maszynowego. Występują w bardzo wielu odmianach i są wykorzystywane w rozwiązywaniu szerokiej gamy problemów. Nawet bardzo pobieżny opis sieci neuronowych wymaga dużo więcej miejsca niż kilka stron przewidzianych w tej pracy. Zainteresowanego czytelnika odsyłam do: [\#REF]x100. (Klasyfikację i opis typów sieci neuronowych można znaleźć w [\#REF].) Tu przedstawione zostaną wyłącznie wybrane zagadnienia mające ścisły związek z pracą, używane będą terminy, których znaczenie wyjaśniane jest w podanych źródłach.

W niniejszej pracy, wykorzystane zostały dwa rodzaje sieci neuronowych: w pełni połączone \angterm{fully connected NN -- FC NN} oraz konwolucyjne \angterm{convolutional NN -- ConvNets, CNN}. 

ToDo: Rys. (własne?) 

opis hiperparametrów (na podstawie obrazka?)

Conv - opis na 2d, zwykle obrazki, tu: 1D np. szeregi czasowe
http://www.deeplearningbook.org/contents/convnets.html - str. 329-334

różnice - lokalne połączenia, dzielenie wag, mniejsza liczba wag w Conv

